%\documentclass[envcountsect,11pt,trans]{beamer}
\documentclass[11pt,aspectratio=169]{beamer}
\input{def}
\mode<presentation>{
    \setbeamertemplate{itemize item}{\color{RedGIZ}$\blacksquare$}
    \setbeamertemplate{itemize subitem}{\color{RedGIZ}$\blacktriangleright$}
		}
\title[DGE--CRED]{Dynamic General Equilibrium Model for Climate Resilient Economic Development}
%\subtitle[]{Training}

\author[Christoph Schult]{Andrej Drygalla, Katja Heinisch and Christoph Schult*} \date[July 2020]{July 2020}
\institute[IWH]{Halle Institute for Economic Research}

 
\logo{\begin{tabular}{c}\includegraphics[keepaspectratio,width=1.00cm]{pictures/seperationbar.png} \\ \includegraphics[keepaspectratio,width=1.00cm]{pictures/GIZ_Logo}\end{tabular}}

\usepackage[style=authoryear,maxbibnames=9,maxcitenames=2,backend=bibtex]{biblatex}
\bibliography{references}

% Settings by yw:
% Outline at beginning of each section:


\AtBeginSection[]{	
	{\setbeamertemplate{footline}{}
	\begin{frame}<beamer>{Outline}
		\tableofcontents[sectionstyle=show/hide, subsectionstyle=show/show/hide, subsubsectionstyle=show/show/hide]
		\setbeamertemplate{footline}{}
		\addtocounter{framenumber}{-1}
	\end{frame}}
}

% Packages:
\usepackage{ragged2e}
\usepackage{upquote}
\usepackage{subfigure}
\usepackage{ragged2e}
\lstset{ 
  backgroundcolor=\color{white},
	breaklines=true,
	basicstyle=\tiny
}
% Section numbering:
\setbeamertemplate{section in toc}[sections numbered]
\setbeamertemplate{subsection in toc}[subsections numbered]
\setbeamertemplate{subsubsection in toc}[subsubsections numbered]
\setbeamertemplate{section in toc}[ball]
\setbeamertemplate{subsection in toc}[ball]
\setbeamertemplate{subsubsection in toc}[ball]
\setbeamercolor{section number projected}{bg={RedGIZ}}
\setbeamercolor{subsection number projected}{bg={RedGIZ}}
\setbeamercolor{subsubsection number projected}{bg={RedGIZ}}
%\usefonttheme{serif}
\begin{document}
%\footnotesize
\usebackgroundtemplate{
\vbox to \paperheight{\vspace{0.1cm}\hbox to \paperwidth{\hfil\includegraphics[width=0.975\paperwidth,height = 0.7\paperheight]{pictures/BackgroundGIZ.jpg}\hfil}\vfil
}}

\begin{frame}<presentation>[noframenumbering,plain]
  \titlepage \\
	{\tiny} \scalebox{.4}{* Research assistance by Yoshiki Wiskamp is greatly acknowledged.}
\end{frame}
\usebackgroundtemplate{
}
{\setbeamertemplate{footline}{}
\begin{frame}<presentation>[noframenumbering]
	\frametitle{Outline}
		 \tableofcontents[hideallsubsections]
\end{frame}
}
%%%%%%%%%%%%%%%%%%%%%%%%%%%%%%%%%%%%%%%%%%%%%%%%%%%%%%%%

\section{Installation of Dynare}

\subsection{Installation of Dynare: Introduction}

\begin{frame}<presentation>
\frametitle{{\thesection.\thesubsection} Installation of Dynare: Introduction}
	\begin{itemize}
		\item Packaged versions of Dynare are available for:
		\begin{itemize}
			\item Windows (7, 8.1, 10)
			\item Several GNU/Linux distributions (Debian, Ubuntu, Linux Mint, Arch Linux)
			\item macOS 10.11 or later
			\item Should work on other systems, but some compilation steps are necessary
		\end{itemize}
	\item In order to run Dynare, you need one of the following:
		\begin{itemize}
			\item MATLAB version 7.9 (R2009b) or above
			\item GNU Octave version 4.2.1 or above, with the statistics package from Octave-Forge
		\end{itemize}
	\item For the DGE CRED model only Dynare 4.6 and the newest Octave version are suitable
	\end{itemize}
\end{frame}


\subsection{Installation of Dynare: Windows}

\begin{frame}<presentation>
\frametitle{{\thesection.\thesubsection} Installation of Dynare: Windows (1)}
	\begin{itemize}
			\item Access the Dynare web page (\texttt{https://www.dynare.org}) and click on ``Download v4.6.1'': 
				\begin{figure}
			\frame{\includegraphics[width=9cm]{pictures/Dynare_installation/windows1}}
		\end{figure}
	\end{itemize}
\end{frame}

\begin{frame}<presentation>
	\frametitle{{\thesection.\thesubsection} Installation of Dynare: Windows (2)}
	\begin{itemize}
		\item The following page will be displayed:
		\begin{figure}
			\frame{\includegraphics[width=9cm]{pictures/Dynare_installation/windows2}}
		\end{figure}
	\item Click on ``Dynare 4.6.1 (exe)'' to download the executable installer.
	\end{itemize}
\end{frame}
\begin{frame}<presentation>
	\frametitle{{\thesection.\thesubsection} Installation of Dynare: Windows (3)}
	\begin{itemize}
		\item Open the downloaded executable installer ``\texttt{dynare-4.6.1.exe}''
		\item Follow the displayed instructions
		\begin{figure}
			\frame{\includegraphics[width=6cm]{pictures/Dynare_installation/windows3}}
		\end{figure}

	\end{itemize}
\end{frame}


\begin{frame}<presentation>
	\frametitle{{\thesection.\thesubsection} Installation of Dynare: Windows (4)}
	\begin{itemize}
		\item Finally, choose a folder in which to install Dynare
		
		\begin{figure}
			\frame{\includegraphics[width=6cm]{pictures/Dynare_installation/windows4}}
		\end{figure}
			\item You will later need to tell MATLAB the path where Dynare is installed
	\end{itemize}
\end{frame}

\subsection{Installation of Dynare: macOS}

\begin{frame}<presentation>
	\frametitle{{\thesection.\thesubsection} Installation of Dynare: macOS (1)}
	\begin{itemize}
		\item Access the Dynare web page (\texttt{https://www.dynare.org}) and click on ``Download v4.6.1'':
		\begin{figure}
			\frame{\includegraphics[width=9cm]{pictures/Dynare_installation/mac1}}
		\end{figure}
	\end{itemize}
\end{frame}
\begin{frame}<presentation>
	\frametitle{{\thesection.\thesubsection} Installation of Dynare: macOS (2)}
	\begin{itemize}
		\item The following page will be displayed:
		\begin{figure}
			\frame{\includegraphics[width=9cm]{pictures/Dynare_installation/mac2}}
		\end{figure}
	\item Click on ``Dynare 4.6.1 (pkg)'' to download the installation package.
	\end{itemize}
\end{frame}
\begin{frame}<presentation>
	\frametitle{{\thesection.\thesubsection} Installation of Dynare: macOS (3)}
	\begin{itemize}
		\item Open the downloaded installation package ``\texttt{dynare-4.6.1.pkg}''
		\item In case that the following security notification is displayed:
		\begin{figure}
			\frame{\includegraphics[width=6cm]{pictures/Dynare_installation/mac3}}
		\end{figure}
		\begin{itemize}
			\item Access: ``System Preferences'' --> ``Security \& Privacy'' --> press: ``Open Anyway''
		\end{itemize}
	\end{itemize}
\end{frame}
\begin{frame}<presentation>
	\frametitle{{\thesection.\thesubsection} Installation of Dynare: macOS (4)}
	\begin{itemize}
		\item The following window should appear:
		\begin{figure}
			\frame{\includegraphics[width=6cm]{pictures/Dynare_installation/mac4}}
		\end{figure}
		\item Follow the displayed instuctions.
	\end{itemize}
\end{frame}
\begin{frame}<presentation>
	\frametitle{{\thesection.\thesubsection} Installation of Dynare: macOS (5)}
	\begin{itemize}
		\item If the installation was successfull the following window will appear:
		\begin{figure}
			\frame{\includegraphics[width=6cm]{pictures/Dynare_installation/mac5}}
		\end{figure}
		\item The Dynare folder should now be listed under ``Applications''.
		\item Dynare is now ready to use and the installation package ``\texttt{dynare-4.6.1.pkg}'' can be deleted.
	\end{itemize}
\end{frame}

\section{Set path to Dynare in Matlab}

\subsection{Set path to Dynare in Matlab: permanently}

\begin{frame}<presentation>
\frametitle{{\thesection.\thesubsection} Set path to Dynare in Matlab: permanently (1)}
	\begin{itemize}
			\item On the MATLAB Home tab, in the Environment section, click on Set Path
				\begin{figure}
			\frame{\includegraphics[width=9cm]{pictures/Dynare_installation/setpath1}}
		\end{figure}
	\end{itemize}
\end{frame}

\begin{frame}<presentation>
\frametitle{{\thesection.\thesubsection} Set path to Dynare in Matlab: permanently (2)}
	\begin{itemize}
			\item Click Add Folder
				\begin{figure}
			\frame{\includegraphics[width=9cm]{pictures/Dynare_installation/setpath2}}
		\end{figure}
	\end{itemize}
\end{frame}

\begin{frame}<presentation>
\frametitle{{\thesection.\thesubsection} Set path to Dynare in Matlab: permanently (3)}
	\begin{itemize}
			\item Select the matlab subdirectory of your Dynare installation
				\begin{figure}
			\frame{\includegraphics[width=9cm]{pictures/Dynare_installation/setpath3}}
		\end{figure}
		\item Then select choose folder
	\end{itemize}
\end{frame}

\begin{frame}<presentation>
\frametitle{{\thesection.\thesubsection} Set path to Dynare in Matlab: permanently (4)}
	\begin{itemize}
			\item Apply the setting by clicking Save button
				\begin{figure}
			\frame{\includegraphics[width=9cm]{pictures/Dynare_installation/setpath4}}
		\end{figure}
	\end{itemize}
\end{frame}

\subsection{Set path to Dynare in Matlab: temporarily}

\begin{frame}<presentation>
\frametitle{{\thesection.\thesubsection} Set path to Dynare in Matlab: temporarily}
	\begin{itemize}
		\item Alternatively, the path to Dynare can be set \textit{temporarily} in Matlab
		\begin{itemize}
			\item This allows to switch between different versions of Dynare
		\end{itemize}
		\item The following command has to be entered in Matlab:
		\begin{itemize}
			\item Windows: \texttt{addpath C:\textbackslash dynare\textbackslash4.6.1\textbackslash matlab}
			\item macOS: \texttt{addpath /Applications/Dynare/4.X.Y/matlab}
		\end{itemize}
	\end{itemize}
\end{frame}


%%%%%%%%%%%%%%%%%%%%%%%%%%%%%%%%%%%%%%%%%%%%%%%%%%%%%%%%

\end{document}
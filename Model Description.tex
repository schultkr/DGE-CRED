\documentclass[10pt,a4paper]{article}
\usepackage{cite}
\usepackage[authoryear]{natbib}     
\usepackage{geometry}
\usepackage{amsmath}
\usepackage{fullpage}
\usepackage{amsfonts}
\usepackage{breqn}
\usepackage{longtable}
\usepackage{graphicx}
\usepackage{hyperref}

\title{Dynamic General Equilibrium Model for Climate Resilient Economic Development (DGE-CRED)\\
\large{Technical Report}}
\date{April 2020}
\author{Christoph Schult and Andrej Drygalla \\ Halle Institute for Economic Research}

% define citation style
\bibliographystyle{agsm}

\begin{document}
\maketitle

\section{Introduction}

Projections  by  the  international  panel  on  climate  change  (IPCC)  show  that  the  global  average  temperature,  the  sea  level  and  the  frequency  of  weather  extremes  are  likely  to  change  as  a  consequence  of  higher  greenhouse  gas  concentration  in  the  atmosphere  \cite{stocker2013climate}.  Climate  change  has  the  potential  to  affect  the  further  economic  development  of  Vietnam.  Previous  studies  \citep[e.g.][]{arndt2015economic,wassmann2004sea,chen2012climate}  show  that  an  increase  in  temperature,  sea  level  rise  and  a  higher  frequency  of  weather  extremes  (e.g.  cyclones  and  droughts)  are  hazards  to  the  future  economic  development.  Vietnam  belongs  to  the  group  of  highly  vulnerable  countries  to  climate  change  

The  impact  of  climate  change  on  Vietnam  will  be  different  across  regions.  Official  statistics  differentiate  between  six  different  statistical  regions  in  Vietnam:  Red  River  Delta,  Northern  Midlands  and  Mountain  Areas  (North  East  and  North  West),  North  Central  and  Central  Coastal  area  (North  Central  Coast  and  South  Central  Coast),  Central  Highlands,  South  East,  and  Mekong  River  Delta.  The  map  in  Figure  \ref{fig:MapVietnam}  shows  that  four  of  the  six  regions  are  located  at  the  coast.  The  impact  of  sea  level  rise  in  Vietnam  will  be  different  for  coastal  and  non-coastal  regions.  For  economic  adaptation  measures  a  regional  dimension  is  important.  

Further, climate change will affect economic activities like agriculture, forestry and fishery differently than manufacturing. Climate variables have different effects on the production factors labour and capital used in the different economic sectors. Adaptation measures can target different sectors and different production factors. Building a damn will reduce the damage by sea level rise on land use. Early warning systems for cyclones are able to reduce the damage on capital. Water management systems are able to tackle the issue of water scarcity due to higher temperatures. Labour productivity will be affected by heatwaves as well. Adaptation measures replacing labour intensive tasks using more capital intensive production processes are potential adaptation measures. To prioritize and evaluate different adaptation measures a cost benefit analysis is necessary. A cost benefit analysis needs to account for the dynamic nature of the problem. It is also necessary to evaluate the sensitivity of the results to different assumptions made in order to get robust policy decisions. Further, the analysis needs to be transparent and all the assumptions made are explicitly stated. Structural mathematical models are a suitable tool for this task. 

Dynamic general equilibrium models with optimizing agents are a standard tool to assess the impact of different policy measures. All adaptation measures will either reduce productivity in the short-run by relocating economic activity or reduce available public funds for other development measures. Therefore, a general equilibrium framework is necessary to assess the economic implications of different adaptation measures. Investment decisions today will affect the future development of specific sectors. This implies path dependency and requires a dynamic framework. We need to differentiate between different regions and economic activities to account for different regional climate developments.

Our model is implemented in the open source environment \href{https://www.dynare.org/}{Dynare} and can be run using Matlab or Octave. An open surce environment allows low costs to acquire the necessary skills and experience to workith the model. The model is calibrated to represent the current economic structure. Sectors in the model correspond to economic activities and the classification by the General Statistical Office (GSO). Regions are based on the statistical regions depicted in Figure \ref{fig:MapVietnam}. It is possible to modify the number of sectors and regions by aggregating the official data. This allows to reduce the size of the model and makes it easier to test new modifications and features of the model. The core of the model can be extended to feature different aspects of the economy. Currently, there are extensions of the model available only for a accessible .

We extend the approach by \cite{nordhaus1993optimal} to model the impact of climate change through damage functions. The link between climate and the economy is modeled using sector and region specific damage functions. Damage functions will affect the productivity of all production factors, or only labour productivity, or the formation of capital. We know that the impact of Vietnamese economic variables on average annual temperature, precipitation, wind speed, average intensity of cyclones and droughts are ´negligible. Our cost beenfit analysis will use the results of metreological models to define paths for climate variables.

We will define different scenarios to evaluate the costs and benefits associated with different climate variables. Therefore, we first need to define a \emph{Baseline} scenario. The Baseline scenario is necessary to define the evolution of the Vietnamese economy without any climate change. Costs associated with climate change are defined as the difference between the Baseline path and another scenario with climate change for any target variable like gross domestic product. The model is a laboratory for policy makers and researchers to conduct experiments by alternating different climate variables and adaption measures. Data is required to set up the experiments and to get plausible results. Reporting the structural equations of the model is equivalent to a protocol and ensures replication of the experiment. Only replication and transparency ensures that the results can be judged by experts and the public. The model simulates long-run developments. The simulations are not meant to predict bumps in the road, e.g. the economic downturn caused by the coronavirus disease 2019 (COVID-19) virus. But, it is possible to include the impact of the COVID-19. One easy way to do this is to use economic forecasts as conducted by the \href{https://www.imf.org/en/Publications/WEO/Issues/2020/04/14/weo-april-2020}{April 2020 World Economic Outlook} by the International Monetary Fund (IMF) to adjust the Baseline scenario to reflect recent developments. Model users are able to quantify upper limits for costs of adaptation measures to reduce damages by climate change. e.g., it is possible to evaluate the impact of temperature increases on different sectors and the overall impact on total gross value added. The discounted cumulative difference between a scenario without a temperature increase and with temperature increase can be used to determine the upper bound for the costs to reduce the damage caused by a temperature increase. 

In the following Section \ref{sec:modelderivation} the derivation of the model equations is explicitly described. Readers who are interested in using the model can skip the model section and can directly go to Section \ref{sec:modelusage}.

\begin{figure}[h]
\caption{Map  of  Vietnam}\label{fig:MapVietnam}
\begin{center}
\includegraphics[width  =  0.5\textwidth,  height  =  0.7\textwidth]{pictures/VietnamStatisticalRegions.jpg}
\end{center}
\footnotesize{Source:  The  illustration  is  published  in  \cite{boateng2012GIS}.}
\end{figure}

\section{Model}\label{sec:modelderivation}
This report is a guide on how to use the spatial small open economy dynamic general equilibrium model for climate change and adaptation simulations. In general the model belongs to the class of real business cycle models, because no nominal rigidities are explicitly considered. Nevertheless, it is possible to extend the model to feature also nominal rigidities. The model structure is depicted in Figure \ref{fig:ModelStructure}.
\begin{figure}[h]
\caption{Model Structure}\label{fig:ModelStructure}
\begin{center}
\includegraphics[width = 1\textwidth, height = 0.7\textwidth]{pictures/ModelStructure.jpg}
\end{center}
\footnotesize{Source: own exhibition.}
\end{figure}
Regional climate variables (precipitation, wind speed, temperature and sea level) are exogenous to economic variables. Regional sectoral production functions depend on regional climate variables. The model is meant to reflect small open economies and therefore the climate system is unaffected by the domestic economic system.

The model consists of an arbitrary number of regions and sectors. Regional differentiation is only provided on the supply side and not on the demand side. Representative households consume sectoral goods and supply capital and labour to the firms in the regions. Households also demand goods and services from the rest of the world. Firms use capital and labour to produce sectoral goods with sectoral and regional specific constant elasticity of substitution production functions.

The government collects taxes, consumes and can use its funds to finance adaptation measures for specific regions and sectors. So far, adaptation measures will reduce overall damage by all climate variables at the same time. The effectiveness of government expenditure in one specific region and sector can vary.


Table \ref{tab:listofsymbols} provides a comprehensive list of all variables and parameters. Appendix \ref{app:modelequationsbaseline} reports all equations of the model.

\subsection{Climate variables}
In order to capture the effect of climate change on the economy it is necessary to include climate variables into the model. A small open economy model does not need to include the impact of domestic economic activity on climate variables. Therefore, in contrast to \cite{nordhaus1993optimal} we do not need to model the interaction between economic activity and climate change. Climate variables are independent of other endogenous variables in the model. We explicitly model the regional average annual temperature $T_{r,t}$, the average precipitation $PREC_{r,t}$, the average annual wind speed $WS_{r,t}$, the sea level $SL_{t}$, cyclones $CYC_{r,t}$ and droughts $DRO_{r,t}$. 
\begin{align}\label{eq:climatevariables}
T_{r,t} = T_{r,0} + \eta_{T,r,t} \nonumber \\
PREC_{r,t} = PREC_{r,0} + \eta_{PREC,r,t} \nonumber \\
W^S_{r,t} = W^S_{r,0} + \eta_{W^S,r,t} \nonumber \\
SL_{t} = SL_{0} + \eta_{SL,r,0} \nonumber \\
CYC_{t} = CYC_{0} + \eta_{CYC,r,0} \nonumber \\
DRO_{t} = SL_{0} + \eta_{DRO,r,0} \nonumber \\
\end{align}

The approach in eq.~\ref{eq:climatevariables} allows to specify the evolution of climate variables according to the projections by meteorological models \citep[e.g.]{stocker2013climate}.

\subsection{Demand}
\subsubsection{Households}
As depicted in Figure \ref{fig:ModelStructure} the demand side is represented by representative households $h$ providing labour $N$ and capital $K$ to domestic firms $f$. Households maximize discounted utility over an infinite horizon by choosing consumption $C_t(h)$, capital $K_{k,r,t+1}(h)$, investments $I_{k,r,t}(h)$, labour $N_{k,r,t}(h)$ and foreign net bond holdings $B_{t+1}$ to maximize utility constrained by the budget constraint and the law of motion for sectoral and regional capital. Therefore, the Lagrangian eq.~\ref{eq:hhlagrangian} of the representative household is
\begin{dmath}\label{eq:hhlagrangian}
\sum_{t=0}^{\infty} \beta^{t} \left[ \left(\frac{C_{t}(h)^{1 - \sigma^{C}}}{1 - \sigma^{C}} - \sum_{k=1}^{K} \sum_{r=1}^{R} \phi^{L}_{k,r} \frac{N_{k,r,t}(h)^{1+\sigma^{L}}}{1+\sigma^{L}} \right) 
- \lambda_{t}(h) \left(P_{t} \, C_{t}(h) \, (1 + \tau^{C}) + \sum_{k=1}^{K} \sum_{r=1}^{R} P_{k,r,t} I_{k,r,t}(h) + S^{f}_{t} \, \phi^{B}_{t} \, (1 + r^{f}_{t} )\, B_{t}(h) - \sum_{k=1}^{K} \sum_{r=1}^{R} (1 - \tau^{N}) \, W_{k,r,t} N_{k,r,t}(h) - \sum_{k=1}^{K} \sum_{r=1}^{R} P_{k,r,t} \, r_{k,r,t} \, (1 - \tau^{K}) \, K_{k,r,t}(h) - B_{t+1}(h) \right) 
- \sum_{k=1}^{K} \sum_{r=1}^{R} \lambda_{t}(h) \omega^{I}_{k,r,t}(h) \left\lbrace K_{k,r,t+1} - (1 - \delta) \, K_{k,r,t} - I_{k,r,t} \, S\left(\frac{I_{k,r,t}}{I_{k,r,t-1}}\right) \right\rbrace  \right].
\end{dmath}
Households receive utility by consuming goods, where the inter temporal elasticity of consumption is defined by $\sigma^{C}$. Dis-utility from labour is sector and region specific $\phi^{L}_{k,r}$, the inverse Frisch elasticity $\sigma^{L}$ is identical for all sectors and regions. Households spent money either on consumption goods $P_t \, C_t(h) \, (1 + \tau^{C})$, regional and sector specific investment $P_{k,r,t} I_{k,r,t}(h)$ and need to repay foreign bonds $B_{t+1}(h)$. They receive income from labour $W_{k,r,t} \, N_{k,r,t}(h) \, (1 - \tau^{L})$, capital renting $P_{k,r,t} \, r_{k,r,t} \, K_{k,r,t}(h) \, (1 - \tau^{K})$ and can use their borrowed money from the foreign economy $B_{t}(h)$. The first order conditions to the problem are the behavioral equations. As is standard in teh literature we replace the Lagrange multiplier $\lambda_{t}$ by the marginal utility of consumption $\frac{C_{t}(h)^{-\sigma^{C}}}{P_{t}\, (1 + \tau^C)}$ derived from the first order condition (FOC) of the above problem with respect to (w.r.t.) consumption. Households supply labour according to the FOC w.r.t. labour eq.~\ref{eq:hhlaboursupply} for each sector and region depending on the wage $W_{k,r,t}$ and the marginal dis-utility of labour for the specific sector and region
\begin{dmath}\label{eq:hhlaboursupply}
\phi^{L}_{k,r} \, A^{N}_{k,r,t} \, N_{k,r,t}(h)^{\sigma^{L}} = \lambda_{t}(h) \, W_{k,r,t} \, (1 - \tau^{N}).
\end{dmath}
The household also needs to decide how much of its income it wants to consume or invest into capital. The famous Euler equation eq.~\ref{eq:hhfoccapital} is obtained by taking the first derivative of the Lagrangian w.r.t. sector and region specific capital
\begin{dmath}\label{eq:hhfoccapital}
\lambda_{t+1}(h) \, \beta \, \left(P_{k,r,t+1} \, r_{k,r,t+1} + (1 - \delta) \, \omega^{I}_{k,r,t+1} \right) = \lambda_{t}(h) \, \omega^{I}_{k,r,t}.
\end{dmath}
Further, the household also faces investment adjustment cost $S(\frac{I_{k,r,t}}{I_{k,r,t-1}}) = 3 - exp\left\lbrace\sqrt{\phi^{K}/2}\left(\frac{I_{k,r,t}}{I_{k,r,t-1}}-1\right\rbrace\right) - exp\left\lbrace-\sqrt{\phi^{K}/2}\left(\frac{I_{k,r,t}}{I_{k,r,t-1}}-1\right)\right\rbrace$, which are sector and region specific. The specification of the investment adjustment cost function is the same as proposed and estimated by \cite{christiano2014risk} for the US.  The marginal value of sectoral and regional investment $\omega^{I}_{k,r,t}$ is determined by
\begin{dmath}\label{eq:hhfocinvestment}
P_{k,r,t} \, \lambda_{t}(h) = \lambda_{t}(h) \, \omega^{I}_{k,r,t} \, \left(S(\frac{I_{k,r,t}}{I_{k,r,t-1}}) - \frac{\partial S(\frac{I_{k,r,t}}{I_{k,r,t-1}})}{\partial I_{k,r,t}} \, \frac{I_{k,r,t}}{I_{k,r,t-1}} \right) + \beta \lambda_{t+1}(h) \, \omega^{I}_{k,r,t+1} \, \frac{\partial S(\frac{I_{k,r,t+1}}{I_{k,r,t}})}{\partial I_{k,r,t}} \, \left(\frac{I_{k,r,t+1}}{I_{k,r,t}}\right)^2
\end{dmath}
Households have access to the international financial market to purchase and sell internationally traded bonds. We only consider net foreign positions.
\begin{dmath}\label{eq:hhfocbonds}
\lambda_{t+1} \, \beta \, S^{f}_{t+1} \, \phi^{B}_{t+1} \left(1+{{r^{f}}_{t+1}}\right) = \lambda_{t} \\
\end{dmath}

The required interest rate will increase if the foreign debt relative to GDP increases and current net exports relative to GDP will decrease. 
\begin{dmath}
\phi^{B}_{t+1} = exp \left(-\phi^B \,(S^{f}_{t+1} \, r^{f}_{t+1} \, \frac{B_{t}}{Y_{t+1}}+\frac{NX_{t}}{Y_t})\right)
\end{dmath}

\subsubsection{Government}

We are interested in different policy measures taken by the government to adapt to a new climate regime. Government behavior is not a result of an optimization problem. The Government collects taxes from consumption $\tau^{C} \, C_{t}$, labour income $\sum_{k}^{K} \sum_{r}^{R} \, (\tau^{N} + \tau_{k,r,t}^{N}) \, W_{k,r,t} \, N_{k,r,t} \, Pop_{t}$ and capital income $\sum_{k}^{K} \sum_{r}^{R} \, (\tau^{K} + \tau_{r,k,t}^{K}) \, P_{k,r,t} \, r_{k,r,t} \, K_{k,r,t}$. In order to finance its activities the government can also get loans from the rest of the world $B^{G}_{t+1}$ and has to repay loans and interest from the previous period denominated in foreign currency $(1 + r^{f}_{t})$ identical to the household. The government budget constraint boils down to eq.~\ref{eq:GovBudget}.
\begin{align}\label{eq:GovBudget}
G_{t} + \sum_{k}^{K} \sum_{r}^{R} G^{A}_{k,r,t} + B^G_{t} =& \sum_{k}^{K} \sum_{r}^{R} \, \left\lbrace (\tau^{K} + \tau_{r,k,t}^{K}) \, P_{k,r,t} \, r_{k,r,t} \, K_{k,r,t} + (\tau^{N} + \tau_{k,r,t}^{N}) \, W_{k,r,t} \, N_{k,r,t} \, Pop_{t} \right\rbrace \nonumber \\
& + (1 + r^{f}_{t}) \, S^{f}_{t} \phi^{B}_{t} \, B^G_{t-1}
\end{align}

Government expenditures can be used to finance adaptation measures in specific sectors and regions $G^{A}_{k,n,t}$. Government expenditures on adaptation measures, taxes on regional and sectoral capital expenditure, and government debt are independent of other variables or to formulate it differently are discretionary. This allows us to evaluate different policy paths for the future and to model the variables by exogenous processes as stated in eq.~\ref{eq:GovExpenditure}.
\begin{align}\label{eq:GovExpenditure}
G^{A}_{k,r,t} = G^{A}_{k,r,0} + \eta^{A}_{k,r,t} \nonumber \\
\tau^{K}_{k,r,t} = \tau^{K}_{k,r,0} + \eta^{\tau^{K}}_{k,r,t} \nonumber \\
\tau^{N}_{k,r,t} = \tau^{N}_{k,r,0} + \eta^{\tau^{N}}_{k,r,t} \nonumber \\
B^G_{t} = B^G_{0} + \eta^{B^{G}}_{t}
\end{align} 


\subsubsection{Resource constraint}

Households and the Government use domestic final goods $Y_t$ produced by firms for consumption, investment and for exports $X_{t}$ and can also use imports $M_t$ for consumption and investment. This gives rise to the well known resource constraint or the expenditure approach to define GDP 
\begin{align}
Y_{t} = C_{t} + I_{t} + G_{t} + \underbrace{X_{t} - M_{t}}_{NX_{t}}
\end{align}

\subsection{Production}

Households demand final domestic goods $Y_{t}$ combining goods from different sectors $Y_{k,t}$ using a CES composition function. They minimize expenditures subject to the composition function
\begin{align}
\underset{Y_{k,t}}{\mathrm{min}} & \sum_{k} Y_{k,t} \, P_{k,t} \\ 
Y_{t} &= \left(\sum_{k} {\omega^{Q}_{k}}^{\frac{1}{\eta^Q}} Y_{k,t}^{\frac{\eta^Q-1}{\eta^Q}} \right)^{\frac{\eta^Q}{\eta^Q-1}}
\end{align}

Therefore demand for sectoral products correspond to the first order conditions of the above optimization problem. The Lagrange multiplier is the price level $P_{t}$ of domestic products. 
\begin{align}
\frac{P_{k,t}}{P_{t}} &= {\omega^{Q}_{k}}^{\frac{1}{\eta^Q}} \left(\frac{Y_{k,t}}{Y_{t}}\right)^{\frac{-1}{\eta^Q}}
\end{align}

In order to model regional economic activity we further decompose the production process on a regional level. One can either think about this approach as modeling the optimization problem of a representative firm operating in one sector on a national level allocating production activity across the nation. Another way is to consider that households make direct purchases from regional operating firms in one sector. In this case the following optimization problem would be part of the above optimization problem. 
\begin{align}
\underset{Y_{k,r,t}}{\mathrm{min}} & \sum_{k} Y_{k,r,t} \, P_{k,r,t} \\ 
Y_{k,t} &= \left(\sum_{k} {\omega^{Q}_{k,r}}^{\frac{1}{\eta^Q_{k}}} Y_{k,r,t}^{\frac{\eta^Q_{k}-1}{\eta^Q_{k}}} \right)^{\frac{\eta^Q_{k}}{\eta^Q_{k}-1}}
\end{align}

Demand for sectoral and regional products correspond to the first order conditions of the above optimization problem. The Lagrange multiplier is the sectoral price level $P_{k,t}$ of domestic products. 
\begin{align}
\frac{P_{k,r,t}}{P_{k,t}} &= {\omega^{Q}_{k,r}}^{\frac{1}{\eta^{Q}_{k}}} \left(\frac{Y_{k,r,t}}{Y_{k,t}}\right)^{\frac{-1}{\eta^{Q}_{k}}}
\end{align}

At the regional and sectoral level are representative firms maximizing profits using capital $K_{k,r,t}$ and labour $L_{k,r,t} = N_{k,r,t} \, Pop_{t}$ provided by households to produce products. They charge a price $P_{k,r,t}$ for their products and have to pay households wages $W_{k,r,t}$, interest on rented capital $P_{r,k,t} \, r_{r,k,t}$, taxes related to the wage bill $\tau^{N}_{r,k,t}$ and on capital expenditure $\tau^{K}_{r,k,t}$.  Representative firms have access to a regional and sector specific constant elasticity of substitution production function. The productivity of capital and labour of a firm in one sector and region depends on the climate variables, and the adaption measures by the government represented by a damage function affecting total factor productivity $A_{k,r,t}$ by $D_{k,r,t} = D_{k,r}\left(T_{r,t}, \, PREC_{r,t}, \, W^{S}_{r,t}, \, SL_{r,t}, \, CYC_{r,t}, \, DRO_{r,t}, \, G^{A}_{r,k,t} \right)$.  Further, we explicity differenitate between climate induced damages affecting labour productivity $D_{N,k,r,t}$ and capital depreciation $D_{K,k,r,t}$. As in \cite{nordhaus1993optimal} we assume a polynomial functional form of the damage functions, but the damages are different across regions and sectors eq.~\ref{eq:Damages}.
\begin{align}
{{D_{k,r}}_{t}}=& exp\left(-\phi^{G^{A}_{k,r}} \, G^{A}_{k,r,t}\right) \, \Big( \nonumber \\
&\underbrace{{{a_{T,1,k,r}}} \, {{T_{r}}_{t}}+{{a_{T,2,k,r}}}\, \left({T_{r}}_{t}\right)^{a_{T,3,k,r}}}_{\mbox{impact of temperature}}+ \underbrace{{{a_{SL,1,k,r}}}\, {{SL}_{t}}+{{a_{SL,2,k,r}}}\, \left({SL}_{t}\right)^{{{a_{SL,3,k,r}}}}}_{\mbox{impact of sea level}} \nonumber \\
+ & \underbrace{{{a_{W^{S},1,k,r}}}\, {{W_{r}^{S}}_{t}}+{{a_{W^{S},2,k,r}}}\, \left({W_{r}^{S}}_{t}\right)^{{{a_{W^{S},3,k,r}}}}}_{\mbox{impact of wind speed}}
+ \underbrace{{{a_{PREC,1,k,r}}} \, {{PREC_{r}}_{t}}+{{a_{PREC,2,k,r}}}\, \left({PREC_{r}}_{t}\right)^{{{a_{PREC,3,k,r}}}}}_{\mbox{impact of precipitation}} \nonumber \\
+ & \underbrace{{{a_{CYC,1,k,r}}}\, {{CYC_{r}}_{t}}+{{a_{CYC,2,k,r}}}\, \left({CYC_{r}}_{t}\right)^{{{a_{CYC,3,k,r}}}}}_{\mbox{impact of cyclones}}
+ \underbrace{{{a_{DRO,1,k,r}}} \, {{DRO_{r}}_{t}}+{{a_{DRO,2,k,r}}}\, \left({DRO_{r}}_{t}\right)^{{{a_{DRO,3,k,r}}}}}_{\mbox{impact of droughts}} \nonumber \\
&\Big). \label{eq:Damages} \\
{{D^{N}_{k,r}}_{t}}=& exp\left(-\phi^{G^{A}_{k,r}} \, G^{A}_{k,r,t}\right) \, \Big( \nonumber \\
&\underbrace{{{a^{N}_{T,1,k,r}}} \, {{T_{r}}_{t}}+{{a^{N}_{T,2,k,r}}}\, \left({T_{r}}_{t}\right)^{a^{N}_{T,3,k,r}}}_{\mbox{impact of temperature}}+ \underbrace{{{a^{N}_{SL,1,k,r}}}\, {{SL}_{t}}+{{a^{N}_{SL,2,k,r}}}\, \left({SL}_{t}\right)^{{{a^{N}_{SL,3,k,r}}}}}_{\mbox{impact of sea level}} \nonumber \\
+ & \underbrace{{{a^{N}_{W^{S},1,k,r}}}\, {{W_{r}^{S}}_{t}}+{{a^{N}_{W^{S},2,k,r}}}\, \left({W_{r}^{S}}_{t}\right)^{{{a^{N}_{W^{S},3,k,r}}}}}_{\mbox{impact of wind speed}}
+ \underbrace{{{a^{N}_{PREC,1,k,r}}} \, {{PREC_{r}}_{t}}+{{a^{N}_{PREC,2,k,r}}}\, \left({PREC_{r}}_{t}\right)^{{{a^{N}_{PREC,3,k,r}}}}}_{\mbox{impact of precipitation}} \nonumber \\
+ & \underbrace{{{a^{N}_{CYC,1,k,r}}}\, {{CYC_{r}}_{t}}+{{a^{N}_{CYC,2,k,r}}}\, \left({CYC_{r}}_{t}\right)^{{{a^{N}_{CYC,3,k,r}}}}}_{\mbox{impact of cyclones}}
+ \underbrace{{{a^{N}_{DRO,1,k,r}}} \, {{DRO_{r}}_{t}}+{{a^{N}_{DRO,2,k,r}}}\, \left({DRO_{r}}_{t}\right)^{{{a^{N}_{DRO,3,k,r}}}}}_{\mbox{impact of droughts}} \nonumber \\
&\Big). \label{eq:DamagesLabour} \\
{{D^{K}_{k,r}}_{t}}=& exp\left(-\phi^{G^{A}_{k,r}} \, G^{A}_{k,r,t}\right) \, \Big( \nonumber \\
&\underbrace{{{a^{K}_{T,1,k,r}}} \, {{T_{r}}_{t}}+{{a^{K}_{T,2,k,r}}}\, \left({T_{r}}_{t}\right)^{a^{K}_{T,3,k,r}}}_{\mbox{impact of temperature}}+ \underbrace{{{a^{K}_{SL,1,k,r}}}\, {{SL}_{t}}+{{a^{K}_{SL,2,k,r}}}\, \left({SL}_{t}\right)^{{{a^{K}_{SL,3,k,r}}}}}_{\mbox{impact of sea level}} \nonumber \\
+ & \underbrace{{{a^{K}_{W^{S},1,k,r}}}\, {{W_{r}^{S}}_{t}}+{{a^{K}_{W^{S},2,k,r}}}\, \left({W_{r}^{S}}_{t}\right)^{{{a^{K}_{W^{S},3,k,r}}}}}_{\mbox{impact of wind speed}}
+ \underbrace{{{a^{K}_{PREC,1,k,r}}} \, {{PREC_{r}}_{t}}+{{a^{K}_{PREC,2,k,r}}}\, \left({PREC_{r}}_{t}\right)^{{{a^{K}_{PREC,3,k,r}}}}}_{\mbox{impact of precipitation}} \nonumber \\
+ & \underbrace{{{a^{K}_{CYC,1,k,r}}}\, {{CYC_{r}}_{t}}+{{a^{K}_{CYC,2,k,r}}}\, \left({CYC_{r}}_{t}\right)^{{{a^{K}_{CYC,3,k,r}}}}}_{\mbox{impact of cyclones}}
+ \underbrace{{{a^{K}_{DRO,1,k,r}}} \, {{DRO_{r}}_{t}}+{{a^{K}_{DRO,2,k,r}}}\, \left({DRO_{r}}_{t}\right)^{{{a^{K}_{DRO,3,k,r}}}}}_{\mbox{impact of droughts}} \nonumber \\
&\Big). \label{eq:DamagesCapital}
\end{align}

Firms in each region and sector have access to a constant elasticity of substitution production function with production factors labour and capital. Eq.~\ref{eq:profitoptim} states the optimization problem of the firm.
\begin{align}\label{eq:profitoptim}
\underset{Y_{k,r,t}, N_{k,r,t}, K_{k,r,t}}{\mathrm{max}} P_{k,r,t} \, Y_{k,r,t} - W_{k,r,t} \, N_{k,r,t} \, Pop_{t} - r_{k,r,t} P_{k,r,t} K_{k,r,t} \nonumber \\ 
\mbox{s.t.} \, Y_{k,r,t} = A_{k,r,t} (1 - D_{k,r,t}) \, \left[{\alpha^{N}_{k,r}}^{\frac{1}{\eta^{NK}_{k,r}}} \, \left( A^{N}_{k,r,t} \, (1 - D^{N}_{k,r,t}) \, Pop_{t} \, N_{k,r,t}\right)^{\rho_{k,r}} + {\alpha^{K}_{k,r}}^{\frac{1}{\eta^{NK}_{k,r}}} \, \left(K_{k,r,t}\right)^{\rho_{k,r}}\right]^{\frac{1}{\rho_{k,r}}}, \nonumber \\
\mbox{ where } \rho_{k,r} = \frac{\eta^{NK}_{k} - 1}{\eta^{NK}_{k}}.
\end{align}

Demand for production factors are given by the first order condition of the above optimization problem. The Lagrange multiplier is equal to the price charged by companies. 

\begin{align}\label{eq:focfirm}
\frac{W_{k,r,t}}{P_{k,r,t}} = {\alpha^{N}_{k,r}}^{\frac{1}{\eta^{NK}_{k,r}}} \, \left(A_{k,r,t} \, (1 - D_{k,r,t}) \, A^N_{k,r,t} \, (1 - D^N_{k,r,t})\right)^{\rho_{k,r}} \left(\frac{Pop_{t} N_{k,r,t}}{Y_{k,r,t}}\right)^{-\frac{1}{\eta^{NK}_{k,r}}} \nonumber \\ 
r_{k,r,t} = {\alpha^{K}_{k,r}}^{\frac{1}{\eta^{NK}_{k,r}}} \, \left(A_{k,r,t} \, (1 - D_{k,r,t})\right)^{\rho_{k,r}}\left(\frac{K_{k,r,t}}{Y_{k,r,t}} \right)^{-\frac{1}{\eta^{NK}_{k,r}}} \\ 
\end{align}

We use the more general case of the CES production function rather than the more commonly used Cobb-Douglas production function. The parameter $\eta^{NK}_{k,r}$ allows us to control the response of capital and labour demand to temporary productivity shocks. Temporary productivity shocks are in our set-up also weather extremes. We will discuss the problem later. 

\subsection{Rest of the world}

The demand for domestic exports and foreign imports is not explicitly modeled in this version of the model. We assume that net exports follow an auto-regressive process of order one and that the long run value of net exports depend on the long-run development of gross domestic product. We therefore assume that imports and exports will grow at the same speed as GDP. Sluggish adjustments in export and import behavior of companies is captured by an auto-regressive process. 
\begin{align}
NX_{t} = \rho^{NX} \, NX_{t-1} + (1 - \rho^{NX}) \omega^{NX} P_{t} \, Y_{t}
\end{align}

The effective exchange rate $S^f_{t}$ and the world interest rate $r^{f}_{t}$ determine how much governments and households have to pay back in domestic currency as net lender or how much they receive as net borrower to the rest of the world. Here the world interest rate is independent of domestic developments and only the effective exchange rate adjusts according to eq.~\ref{eq:hhfocbonds}.

\subsection{Model extension: intermediate consumption and sectoral trade} 
Table \ref{tab:listofsymbolsextension1} provides a list of all additional variables and parameters included for the extension. Appendix \ref{app:modelequationsextension1} reports all additional and modified equations associated with the respective extension.

In order to consider the impact of climate change on exports and imports, we need to extend the model to feature sectoral exports $X_{k,t}$ and imports $M_{k,t}$. It is also necessary to include intermediate consumption $Q^{I}_{k,r,t}$ by firms to ensure that the model can be calibrated to match export and import ratios for different sectors. In contrast to the baseline version imports are used to produce final domestically used output $Q^U_t$. Final domestically used output is provided by a representative firm using domestic output from different sectors and regions $Q^{D}_{k,t}$ and sectoral imports. Imports and domestic products from different sectors are combined using a CES composition function. The production function of the firm is a nested CES function.
\begin{align}
Q^U_t = \left((1 - \omega^{F})^{\frac{1}{\eta^{F}}} \, {Q^D_t}^{\rho^F} + {\omega^{F}}^{\frac{1}{\eta^{F}}} \, {M_t}^{\rho^F}\right)^{\frac{1}{\rho^F}}, \mbox{where } \rho^{F} = \frac{\eta^F - 1}{\eta^F} \label{eq:cesfinal}\\
M_t = \left(\sum_{k}^{K}{\omega^{M}_{k}}^{\frac{1}{\eta^{M}}} \, {M_{k,t}}^{\rho^M}\right)^{\frac{1}{\rho^M}}, \mbox{where } \rho^{M} = \frac{\eta^M - 1}{\eta^M} \label{eq:imports} \\
Q^D_t = \left(\sum_{k}^{K}{\omega^{Q}_{k}}^{\frac{1}{\eta^{Q}}} \, {Q^D_{k,t}}^{\rho^Q}\right)^{\frac{1}{\rho^Q}}, \mbox{where } \rho^{Q} = \frac{\eta^Q - 1}{\eta^Q} \label{eq:domesticusedoutput} \\
Q^{D}_{k} = Q_{k,t} - X_{k,t} \nonumber \\ 
Q_{k,t} = \left(\sum_{k}^{K}{\omega^{Q}_{k,r}}^{\frac{1}{\eta^{Q}_{k}}} \, {Q_{k,r,t}}^{\rho^Q_{k}}\right)^{\frac{1}{\rho^Q_{k}}}, \mbox{where } \rho^{Q}_{k} = \frac{\eta^Q_{k} - 1}{\eta^Q_{k}} \nonumber 
\end{align}

Prices of sectoral imports $P^{M}_{k,t}$ are exogenous eq.~\eqref{eq:importprices} as well as the export demand eq.~\eqref{eq:exportsdemand} for domestic produced sectoral products. The optimization problem of the final goods producer can be split up into different parts. At the top level eq.~\eqref{eq:cesfinal} they choose the amount of imports according to eq.~\eqref{eq:demandimports} and domestically used and produced products eq.~\eqref{eq:demanddomesticoutput}. In the next layer the final goods producer minimizes costs for a given level of imports eq.~\eqref{eq:imports} and domestically used and produced output eq.~\eqref{eq:domesticusedoutput}. This leads to demand functions for sectoral domestic output eq.~\eqref{eq:demandsectoraloutput} and sectoral imports eq.~\eqref{eq:demandsectoralimports}.

Further, imports and domestic production are used to consume, invest and for intermediate goods $Q^{I}_{k,r,t}$ in the domestic production process. Therefore, firms are using labour and capital, but also intermediate input to produce goods and services. The production function of firms operating in different regions is now modified to feature another layer of the CES production function. The firm is now maximizing profits combining value-added $Y_{k,r,t}$ and intermediate products $Q^{I}_{k,r,t}$ to produce output $Q_{k,r,t}$. They sell their products under prefect competition for the price $P^{D}_{k,r,t}$. The optimization problem is 
\begin{align}
\underset{Q^{I}_{k,r,t}, \, Y_{k,r,t}}{\mathrm{min}} P^{D}_{k,r,t} \, Q_{k,r,t} - P_{t} \, Q^{I}_{k,r,t} - P_{k,r,t} \, Y_{k,r,t} \nonumber \\
Q_{k,r,t} = \left({\omega^{Q^{I}}_{k,r}}^{\frac{1}{\eta^{I}_{k,r}}}  \, {Q^{I}_{k,r,t}}^{rho^{I}_{k,r}} + (1 - \omega^{Q^{I}}_{k,r})^{\frac{1}{\eta^{I}_{k,r}}} \, {Y_{k,r,t}}^{\rho^{I}_{k,r}} \right)^{\frac{1}{\rho^I_{k,r}}}, \mbox{where } \rho^I_{k,r} = \frac{\eta^{I}_{k,r} - 1}{\eta^{I}_{k,r}}
\end{align}

Therefore, we obtain equations determining the demand for value added eq.~\eqref{eq:demandva} and intermediate products eq.~\eqref{eq:demandinput} by one sector in a region depending on the relative prices of labour, capital and intermediate products.
\begin{align}
\frac{{{P_{k,r}}_{t}}}{{P^{D}_{k,r,t}}}=\left(1 - {\omega^{Q^{I}}_{k,r}}\right)^{\frac{1}{{{\eta^{I}_{k}}}}} \, \left(\frac{Y_{k,r,t}}{Q_{k,r,t}}\right)^{\frac{\left(-1\right)}{{{\eta^{I}_{k}}}}} \label{eq:demandva}\\
\frac{{P_{t}}}{{P^{D}_{k,r,t}}}={\omega^{Q^{I}}_{k,r}}^{\frac{1}{{{\eta^{I}_{k}}}}} \, \left(\frac{Q^{I}_{k,r,t}}{Q_{k,r,t}}\right)^{\frac{\left(-1\right)}{{{\eta^{I}_{k}}}}} \label{eq:demandinput}
\end{align}

Net exports eq.~\eqref{eq:netexports} are now the difference between aggregate export expenditures $P^{D}_{t} X_{t}$ and aggregate import expenditures $P^{M}_{t} M_{t}$. The resource constraint now equates total domestic production with domestic consumption, investment, government consumption, net exports and intermediate products eq.~\eqref{eq:resourceconstraint}. Gross value added is total domestic output less total intermediate consumption.

\subsection{Model extension: energy consumption} 
Table \ref{tab:listofsymbolsextension2} provides a list of all additional variables and parameters included for the extension. Appendix \ref{app:modelequationsextension2} reports all additional and modified equations associated with the respective extension.

To model the impact of mitigation policy on the economy we define an energy sector $k^{E}$. We add two new layers to the production function in the energy sector. The energy sector uses intermediate products $Q^{I}_{k^{E},r,t}$ and a composite energy product $E_{k,r,t}$ to produce a composite intermediate product $Q^{I,E}_{k,r,t}$ eq.~\eqref{eq:energyintermediatecomposite}. The composite energy input is a combination of different energy carriers $c \in \{1, \dots, EC\}$ using a CES production function eq.~\eqref{eq:energycomposite}. The price of an energy carrier is the same across regions and exogenous to other variables eq.~\eqref{eq:priceenergycarrier}. Aggregate consumption of energy carriers eq.~\eqref{eq:energycarrieraggregate} is necessary to compute emissions associated with different energy carriers. Total  intermediate input $Q^{I}_{t}$ expenditures is the sum over all intermediate inputs demanded by different sectors including expenditures on energy carriers eq.~\eqref{eq:aggregationintermediateinputswithenergy}.

%\section{Scenario Analysis}
%
%It is necessary to define different Scenarios in order to quantify the economic costs of climate change on the Vietnamese economy. We follow here the approach by \cite{}



\cleardoublepage
\section{How to use the model?}\label{sec:modelusage}
\subsection{Usage}
\begin{enumerate}
\item In order to use the model you need to install \href{https://www.dynare.org/download/}{Dynare (at least version 4.6.1)}  and \href{https://www.mathworks.com/products/matlab.html}{Matlab (at least 2018b)} or \href{https://www.gnu.org/software/octave/}{Octave} on your computing machine. 
For Octave you need to have the version 5.2.0 as reported by the Dynare team. 
\item You need to download the repository from Github. 
\item Open Octave or Matlab GUI and browse to the location of the folder in your computer. You have the right folder if the command {\tt pwd()} returns {\tt YourPath/DGE-CRED/DGE_CRED_Model}.
\item The script {\tt RunSimulations.m} has to be executed in order to run simulations for different scenarios. Make sure that the scenarios and model parameters are defined in the file \\ {\tt ModelSimulationandCalibrationKSectorsandRRegions.xlsx}. We need to adopt the number of sectors and regions in the file {\tt DGE\_CRED\_Model.mod}.
\item The simulation results are stored in the file {\tt ResultsScenariosKSectorsandRRegions.xlsx}.
\end{enumerate}

\section{Folder structure}
\begin{enumerate}
\item The main file containing all necessary mod files is {\tt DGE_CRED_Model.mod}. This file includes the following files stored in the {\tt ModFiles} folder:
\begin{enumerate}
\item {\tt DGE_CRED_Model_Declarations.mod} declares all endogenous and exogenous variables if the model and structural parameters.
\item {\tt DGE_CRED_Model_Parameters.mod} assigns values to the structural parameters of the model.
\item {\tt DGE_CRED_Model_Equations.mod} contains the equations of the model.
\item {\tt DGE_CRED_Model_LatexOutput.mod} produces latex output for documentation of the declared variables and model equations.
\item {\tt DGE_CRED_Model_SteadyState.mod} computes initial and terminal condition for the dynamic simulation.
\item {\tt DGE_CRED_Model_Simulations.mod} starts the dynamic simulation.
\end{enumerate}
\item Subroutines responsible for finding the initial and terminal conditions are located in the subfolder {\tt Functions}:
\begin{enumerate}
\item {\tt Calibration.mat} finds the initial conditions to reflect a specific year of the economy.
\item {\tt FindA.mat} looks for exogenous productivity shocks across sectors and regions to meet the terminal conditions.
\item {\tt FindK.mat} looks for a capital allocation across sectors and regions to fulfill the static equations of the model.
\item {\tt rng.mat} random number generator function necessary for Octave users.
\item {\tt LoadExogenous.mat} reads exogenous variables for different scenarios.
\end{enumerate}
\item To define scenarios and structural parameters one needs to create an Excel workbook located in the subfolder {\tt ExcelFiles}:
\begin{enumerate}
\item {\tt ModelSimulationandCalibrationforKSectorsandRregions.xlsx} has multiple sheets:
\begin{enumerate}
\item initial {\tt Start}
\item terminal {\tt Terminal}
\item parameters to define rigidity parameters {\tt Dynamics}
\item elasticity parameters and tax rates {\tt Structural Parameters}
\item coefficients for regional and sector specific damage functions {\tt Climate Damage Functions}
\item {\tt Baseline} scenario and other optional scenario sheets defining paths for exogenous varibales
\end{enumerate}
\item {\tt ResultsScenariosKSectorsandRregions.xlsx} has as many sheets as Scenarios defined in the previous Excel file.
\end{enumerate}
\item The latex files produced by {\tt DGE_CRED_Model_LatexOutput.mod} are stored in {\tt LatexFiles}.
\begin{enumerate}
\item the system of dynamic equations as implemented in Matlab {\tt DGE_CRED_Model_Dynamic}, \\
 {\tt DGE_CRED_Model_Dynamic_content}
\item names of endogenous, exogenous variables and parameters {\tt DGE_CRED_Model_latex_definitions}
\item the system of dynamic equations in original form without auxiliary variables for leads and lags {\tt DGE_CRED_Model_original}, {\tt DGE_CRED_Model_original_content}
\end{enumerate}
\item The file to run different simulations is {\tt RunSimulations.m}.
\item A Matlab function to find solutions to the static system of equations is {\tt DGE_CRED_Model_steady_state.m}.
\end{enumerate}

\bibliography{references}
\cleardoublepage
\appendix
\section{Model equations for baseline version}\label{app:modelequationsbaseline}
\footnotesize
% Equation 1
\subsection{Regional Industries}
% Equation 1
damage function TFP
\begin{dmath}
{{D_{k,r}}_{t}}=\left\lbrace\left({{a_{T,1,k,r}}} \, {{T_{r}}_{t}}+{{a_{T,2,k,r}}}\, \left({T_{r}}_{t}\right)^{a_{T,3,k,r}}\right) \, exp\left(-\phi^{G^{A,T}_{k,r}} \, G^{A,T}_{k,r,t}\right)
 + \left({{a_{SL,1,k,r}}}\, {{SL}_{t}}+{{a_{SL,2,k,r}}}\, \left({SL}_{t}\right)^{{{a_{SL,3,k,r}}}}\right)  \, I\left(SL_{t} > \frac{K^{A,SL}_{k,r,t}}{\phi^{G^{A,SL}_{k,r}}} \right)
 + \left({{a_{W,1,k,r}}}\, {{WS_{r}}_{t}} + {{a_{W,2,k,r}}}\, \left({WS_{r}}_{t}\right)^{{{a_{W,3,k,r}}}} \right) \, exp\left(-\phi^{G^{A,WS}_{k,r}} \, G^{A,WS}_{k,r,t}\right)
 + \left({{a_{P,1,k,r}}}\, {{PREC_{r}}_{t}}+{{a_{P,2,k,r}}}\, \left({PREC_{r}}_{t}\right)^{{{a_{P,3,k,r}}}}\right) \, exp\left(-\phi^{G^{A,PREC}_{k,r}} \, G^{A,PREC}_{k,r,t}\right)
 + \left({{a_{C,1,k,r}}}\, {{CYC_{r}}_{t}}+{{a_{C,2,k,r}}}\, \left({CYC_{r}}_{t}\right)^{{{a_{C,3,k,r}}}}\right) \, exp\left(-\phi^{G^{A,CYC}_{k,r}} \, G^{A,CYC}_{k,r,t}\right)
 + \left({{a_{D,1,k,r}}}\, {{DRO_{r}}_{t}}+{{a_{D,2,k,r}}}\, \left({DRO_{r}}_{t}\right)^{{{a_{DRO,3,k,r}}}}\right) \, exp\left(-\phi^{G^{A,DRO}_{k,r}} \, G^{A,DRO}_{k,r,t}\right)
\right\rbrace
\end{dmath}
damage function capital
\begin{dmath}
{{D^{K}_{k,r}}_{t}}=\left({{a^{K}_{T,1,k,r}}} \, {{T_{r}}_{t}}+{{a^{K}_{T,2,k,r}}}\, \left({T_{r}}_{t}\right)^{a^{K}_{T,3,k,r}}+{{a^{K}_{SL,1,k,r}}}\, {{SL}_{t}}+{{a^{K}_{SL,2,k,r}}}\, \left({SL}_{t}\right)^{{{a^{K}_{SL,3,k,r}}}}+{{a^{K}_{W,1,k,r}}}\, {{WS_{r}}_{t}}+{{a^{K}_{W,2,k,r}}}\, \left({WS_{r}}_{t}\right)^{{{a^{K}_{W,3,k,r}}}}+{{a^{K}_{P,1,k,r}}}\, {{PREC_{r}}_{t}}+{{a^{K}_{P,2,k,r}}}\, \left({PREC_{r}}_{t}\right)^{{{a^{K}_{P,3,k,r}}}}+{{a^{K}_{C,1,k,r}}}\, {{CYC_{r}}_{t}}+{{a^{K}_{C,2,k,r}}}\, \left({CYC_{r}}_{t}\right)^{{{a^{K}_{C,3,k,r}}}}+{{a^{K}_{D,1,k,r}}}\, {{DRO_{r}}_{t}}+{{a^{K}_{D,2,k,r}}}\, \left({DRO_{r}}_{t}\right)^{{{a^{K}_{DRO,3,k,r}}}}\right) 
\end{dmath}
damage function labour productivity
\begin{dmath}
{{D^{N}_{k,r}}_{t}}=\left({{a^{N}_{T,1,k,r}}} \, {{T_{r}}_{t}}+{{a^{N}_{T,2,k,r}}}\, \left({T_{r}}_{t}\right)^{a^{N}_{T,3,k,r}}+{{a^{N}_{SL,1,k,r}}}\, {{SL}_{t}}+{{a^{N}_{SL,2,k,r}}}\, \left({SL}_{t}\right)^{{{a^{N}_{SL,3,k,r}}}}+{{a^{N}_{W,1,k,r}}}\, {{WS_{r}}_{t}}+{{a^{N}_{W,2,k,r}}}\, \left({WS_{r}}_{t}\right)^{{{a^{N}_{W,3,k,r}}}}+{{a^{N}_{P,1,k,r}}}\, {{PREC_{r}}_{t}}+{{a^{N}_{P,2,k,r}}}\, \left({PREC_{r}}_{t}\right)^{{{a^{N}_{P,3,k,r}}}}+{{a^{N}_{C,1,k,r}}}\, {{CYC_{r}}_{t}}+{{a^{N}_{C,2,k,r}}}\, \left({CYC_{r}}_{t}\right)^{{{a^{N}_{C,3,k,r}}}}+{{a^{N}_{D,1,k,r}}}\, {{DRO_{r}}_{t}}+{{a^{N}_{D,2,k,r}}}\, \left({DRO_{r}}_{t}\right)^{{{a^{N}_{DRO,3,k,r}}}}\right) 
\end{dmath}
capital stock financed by government expenditure for adaptation measures against against different climate variables $z \in \{T,\, WS,\, PREC, \, DRO,\, CYC,\, SL\}$
\begin{dmath}
K^{A,z}_{k,r,t+1}=\eta^{A,z}_{k,r,t}
\end{dmath}
law of motion of capital stock financed by government expenditure
\begin{dmath}
K^{A,z}_{k,r,t+1}= (1 - \delta_{K^{A,z},k,r}) K^{A,z}_{k,r,t} + G^{A,z}_{k,r,t}
\end{dmath}
% Equation 2
TFP
\begin{dmath}
A_{k,r,t}= A_{k,r,0} \, exp\left({\eta_{A,k,r,t}}\right)
\end{dmath}
% Equation 3
%capital specific productivity
%\begin{dmath}
%A^{K}_{k,r,t}= A^{K}_{k,r,0} \, exp\left({\eta_{A^{K},k,r,t}}\right)
%\end{dmath}
% Equation 4
labour specific productivity
\begin{dmath}
A^{N}_{k,r,t}= A^{N}_{k,r,0} \, exp\left({\eta_{A^{N},k,r,t}}\right)
\end{dmath}
% Equation 3
price of regional sectoral goods
\begin{dmath}
\frac{{{P_{k,r}}_{t}}}{{{P_k}_{t}}}={{\omega^{Q}_{k,r}}}^{\frac{1}{{{\eta^{Q}_{k}}}}}\, \left(\frac{{{Y_{k,r}}_{t}}}{{{Y_k}_{t}}}\right)^{\frac{\left(-1\right)}{{{\eta^{Q}_{k}}}}}
\end{dmath}
% Equation 3
production function
\begin{dmath}
{{Y_{k,r}}_{t}}={{A_{k,r}}_{t}}\, \left(1-{{D_{k,r}}_{t}}\right)\, \left({{\alpha^{K}_{k,r}}}^{\frac{1}{{{\eta^{N,K}_{k,r}}}}}\, \left({{K_{k,r}}_{t}}\right)^{\frac{{{\eta^{N,K}_{k,r}}}-1}{{{\eta^{N,K}_{k,r}}}}}+{{\alpha^{N}_{k,r}}}^{\frac{1}{{{\eta^{N,K}_{k,r}}}}}\, \left({{A^{N}_{k,r}}_{t}} \, (1 - D^{N}_{k,r,t}) \, (1 -  {Pop_{t}}\, {{N_{k,r}}_{t}}\right)^{\frac{{{\eta^{N,K}_{k,r}}}-1}{{{\eta^{N,K}_{k,r}}}}}\right)^{\frac{{{\eta^{N,K}_{k,r}}}}{{{\eta^{N,K}_{k,r}}}-1}}
\end{dmath}
% Equation 4
firms FOC capital
\begin{dmath}
{{r_{k,r}}_{t}} \, \left(1+\tau^{K,F}_{k,r,t}\right)={{\alpha^{K}_{k,r}}}^{\frac{1}{{{\eta^{N,K}_{k,r}}}}}\, \left({A_{k,r}}_{t} \, (1 - D_{k,r,t})\right)^{\frac{{{\eta^{N,K}_{k,r}}}-1}{{{\eta^{N,K}_{k,r}}}}}\, \left(\frac{{{K_{k,r}}_{t}}}{{{Y_{k,r}}_{t}}}\right)^{\frac{-1}{{{\eta^{N,K}_{k,r}}}}}
\end{dmath}
% Equation 5
firms FOC labour
\begin{dmath}
\frac{{{W_{k,r}}_{t}}\left(1+\tau^{N,F}_{k,r,t}\right)}{{{P_{k,r}}_{t}}}={{\alpha^{N}_{k,r}}}^{\frac{1}{{{\eta^{N,K}_{k,r}}}}}\, \left(A_{k,r,t} \, (1 - D_{k,r,t}) \, A^{N}_{k,r,t} \, (1 - D^{N}_{k,r,t})\right)^{\frac{\eta^{N,K}_{k,r}-1}{\eta^{N,K}_{k,r}}} \, \left(\frac{{Pop_{t}}\, {{N_{k,r}}_{t}}}{{{Y_{k,r}}_{t}}}\right)^{\frac{-1}{{{\eta^{N,K}_{k,r}}}}}
\end{dmath}

\subsection{Aggregation}
relative price of sectoral output
\begin{dmath}
\frac{{{P_k}_{t}}}{{P_{t}}}={{\omega^{Q}_{k}}}^{\frac{1}{{{\eta^{Q}}}}}\, \left(\frac{{{Y_k}_{t}}}{{Y_{t}}}\right)^{\frac{\left(-1\right)}{{{\eta^{Q}}}}}
\end{dmath}
sectoral CES aggregation
\begin{dmath}
Y_{k,t}=\left(\sum_{r}^{R}{{\omega^{Q}_{k,r}}}^{\frac{1}{{{\eta^{Q}_{k}}}}}\, {{Y_{k,r}}_{t}}^{\frac{{{\eta^{Q}_{k}}}-1}{{{\eta^{Q}_{k}}}}}\right)^{\frac{{{\eta^{Q}_{k}}}}{{{\eta^{Q}_{k}}}-1}}
\end{dmath}


\subsection{Households}
% Equation 6
households FOC labour
\begin{dmath}
\frac{{{W_{k,r}}_{t}}\, \left(1-{{\tau^{N}}}\right)\, \left(\frac{{C_{t}}}{{Pop_{t}}}\right)^{\left(-{{\sigma^{C}}}\right)}}{\left(1+{{\tau^{C}_{t}}}\right) \, P_{t}}={{\phi^{L}}}\, {{N_k}_{t}}^{{{\sigma^{L}}}}
\end{dmath}
% Equation 7
households FOC capital
\begin{dmath}
\frac{\left(\frac{{P_{t+1}}\, {C_{t+1}}}{{Pop_{t+1}}}\right)^{\left(-{{\sigma^{C}}}\right)}}{\left(1+{{\tau^{C}_{t}}}\right) \, P_{t+1}}\, {{\beta}}\, {{P_{k,r}}_{t+1}}\, {{r_{k,r}}_{t+1}}\, \left(1-{{\tau^{K}}}\right)+{{\beta}}\, {{\omega^I_{k,r}}_{t+1}}\, \left(1-{{\delta}} - D^{K}_{k,r,t+1}\right)={{\omega^I_{k,r}}_{t}}
\end{dmath}
% Equation 8
households FOC investment
\begin{dmath}
P_{k,r,t}\, \frac{\left(\frac{{C}_{t}}{{Pop}_{t}}\right)^{\left(-{{\sigma^{C}}}\right)}}{{P}_{t}\, \left(1+{{\tau^{C}_{t}}}\right)}={{\omega^I_{k,r}}}_{t} \, \frac{\left(\frac{{C}_{t}}{{Pop}_{t}}\right)^{\left(-{{\sigma^{C}}}\right)}}{{P}_{t}\, \left(1+{{\tau^{C}_{t}}}\right)}\, \left(\Gamma\left(\frac{{{I_{k,r}}}_{t}}{{{I_{k,r}}}_{t-1}}\right) - \Gamma'\left(\frac{{{I_{k,r}}}_{t}}{{{I_{k,r}}}_{t-1}}\right) \, \left(\frac{{{I_{k,r}}}_{t}}{{{I_{k,r}}}_{t-1}}\right) \right) + {{\omega^I_{k,r}}}_{t+1}\, \frac{\left(\frac{{C}_{t+1}}{{Pop}_{t+1}}\right)^{\left(-{{\sigma^{C}}}\right)}\, {{\beta}}}{\left(1+{{\tau^{C}_{t}}}\right)\, {P}_{t+1}} \, \Gamma'\left(\frac{{{I_{k,r}}}_{t+1}}{{{I_{k,r}}}_{t}}\right) \, \frac{{{I_{k,r}}}_{t+1}^{2}}{{{I_{k,r}}}_{t}^{2}}
\end{dmath}
% Equation 8
households LOM capital
\begin{dmath}
{{K_{k,r}}_{t+1}}={{K_{k,r}}_{t}}\, \left(1-{{\delta}}-D^K_{k,r,t}\right) + I_{k,r,t} \, \Gamma\left(\frac{{{I_{k,r}}_{t}}}{{{I_{k,r}}_{t-1}}}\right)
\end{dmath}
households FOC foreign bonds
\begin{dmath}
\frac{\left(\frac{{C}_{t+1}}{{Pop}_{t+1}}\right)^{\left(-{{\sigma^{C}}}\right)}}{\left(1+{{\tau^{C}_{t}}}\right)\, {P}_{t+1}}\, {{\beta}}\, {S^{f}}_{t+1}\, \exp\left(-\phi^{B}\, \left(\frac{{B}_{t+1}\, {S^{f}}_{t+1}\, {{r^{f}}}_{t+1}}{{Y}_{t+1}}+\frac{{NX}_{t+1}}{{Y}_{t+1}}\right)\right)\, \left(1+{{r^{f}}}_{t+1}\right)=\frac{\left(\frac{{C}_{t}}{{Pop}_{t}}\right)^{\left(-{{\sigma^{C}}}\right)}}{{P}_{t}\, \left(1+{{\tau^{C}_{t}}}\right)}
\end{dmath}

\subsection{Climate Variables}
temperature
\begin{dmath}
{{T_{r}}_{t}}={{T_{0,r}}}+{{\eta_{T,r}}_{t}}
\end{dmath}
wind speed
\begin{dmath}
{{WS_{r}}_{t}}={{WS_{0,r}}}+{{\eta_{WS,r}}_{t}}
\end{dmath}
precipitation
\begin{dmath}
{{PREC_{r}}_{t}}={{PREC_{0,r}}}+{{\eta_{PREC,r}}_{t}}
\end{dmath}
sea level
\begin{dmath}
{{SL}_{t}}={{SL_0}}+{{\eta_{SL}}_{t}}
\end{dmath}
drought
\begin{dmath}
{{DRO}_{t}}={{DRO_0}}+{{\eta_{DRO}}_{t}}
\end{dmath}
cyclone
\begin{dmath}
{{CYC}_{t}}={{CYC_0}}+{{\eta_{CYC}}_{t}}
\end{dmath}

\subsection{Trade}
trade balance
\begin{dmath}
{NX_{t}}=-\left({B_{t+1}}-\left(1+{{r^{f}}_{t}}\right) \, S^{f}_{t} \, {B_{t}}\right)
\end{dmath}
net exports
\begin{dmath}
{NX_{t}}={{\rho^{NX}}}\, {NX_{t-1}}+{P_{t} \, Y_{t}}\, \left(1-{{\rho^{NX}}}\right)\, exp\left({{\eta_{NX}}_{t}}\right)\, {{\omega^{NX}}}
\end{dmath}
foreign interest rates
\begin{dmath}
r^{f}_{t} = r^{f}_{0} + \eta^{r^{f}}_t
\end{dmath}

\subsection{Government}
budget constraint with adaptation measures for climate variables $z \in \{T,\, SL,\, WS,\, PREC,\, CYC,\, DRO\}$
\begin{dmath}
{P_{t}}\, {G_{t}} + \sum_{r}^{R} \sum_{k}^{K} {P_{t}} \, \sum_{z} G^{A,z}_{k,r,t} + {P_{t}}\, {B^{G}_{t+1}}={P_{t}} \, {S^{f}_{t}} \, \left(1+{{r^{f}}_{t}}\right)\, {B^{G}_{t}}+{C_{t}}\, {P_{t}}\, {{\tau^{C}_{t}}}+\sum_{k}^{K} \sum_{r}^{R} \, N_{k,r,t} \, W_{k,r,t} \, \left({\tau^{N,H}_{t} + \tau^{N,F}_{k,r,t}}\right)+{{K_{k,r}}_{t}}\, {{r_{k,r}}_{t}}\, {{P_{k,r}}_{t}}\, \left(\tau^{K,H}_{t} + \tau^{K,F}_{k,r,t}\right)
\end{dmath}
government foreign debt
\begin{dmath}
{B^{G}_{t}}={{\eta_{B^{G}}}_{t}}
\end{dmath}
tax rates on capital expenditure firms
\begin{dmath}
\tau^{K,F}_{k,r,t} = \tau^{K,F}_{k,r,0} + \eta^{\tau^{K,F}}_{k,r,t}
\end{dmath}
tax rates on labour compensation firms
\begin{dmath}
\tau^{N,F}_{k,r,t} = \tau^{N,F}_{k,r,0} + \eta^{\tau^{N,F}}_{k,r,t}
\end{dmath}
tax rates on capital expenditure households
\begin{dmath}
\tau^{K,H}_{k,t} = \tau^{K,H}_{r,0} + \eta^{\tau^{K,H}}_{r,t}
\end{dmath}
tax rates on labour compensation households
\begin{dmath}
\tau^{N,H}_{r,t} = \tau^{N,H}_{r,0} + \eta^{\tau^{N,H}}_{r,t}
\end{dmath}



\subsection{Aggregates}
national price level
\begin{dmath}
{P_{t}}=exp\left({{\eta_{P}}_{t}}\right)
\end{dmath}
national population
\begin{dmath}
Pop_{t}=Pop_0 + \eta_{Pop,t}
\end{dmath}
resource constraint
\begin{dmath}
{Y_{t}}={C_{t}}+{I_{t}}+{G_{t}}+\sum_{k}^{K} \sum_{r}^{R} {G^{A}_{k,r,t}} + {NX_{t}}
\end{dmath}
sector labour
\begin{dmath}
{{N_k}_{t}}={\sum_{r}^{R} {N_{k,r}}_{t}}
\end{dmath}
sector wage bill
\begin{dmath}
{{N_k}_{t}}\, {{W_k}_{t}}={\sum_{r}^{R} {N_{k,r}}_{t}}\, {{W_{k,r}}_{t}}
\end{dmath}
sector investment
\begin{dmath}
{{P_k}_{t}}\, {{I_k}_{t}}={\sum_{r}^{R} {P_{k,r}}_{t}}\, {{I_{k,r}}_{t}}
\end{dmath}
sector capital stock
\begin{dmath}
{{P_k}_{t}}\, {{K_k}_{t}}={\sum_{r}^{R} {P_{k,r}}_{t}}\, {{K_{k,r}}_{t}}
\end{dmath}
national investment
\begin{dmath}
{P_{t}}\, {I_{t}}={\sum_{k}^{K} {P_k}_{t}}\, {{I_k}_{t}}
\end{dmath}
national capital
\begin{dmath}
{P_{t}}\, {K_{t}}={\sum_{k}^{K} {P_k}_{t}}\, {{K_k}_{t-1}}
\end{dmath}
national output
\begin{dmath}
{P_{t}}\, {Y_{t}}={\sum_{k}^{K} {P_k}_{t}}\, {{Y_k}_{t}}
\end{dmath}
national labour share
\begin{dmath}
{N_{t}}={\sum_{k}^{K} {N_k}_{t}}
\end{dmath}

\section{Additional and modified model equations for input-output extension}\label{app:modelequationsextension1}
\footnotesize
% Equation 1
\subsection{Regional Industries}
% Equation 3
demand for regional sector output
\begin{dmath}
\frac{{{P^{D}_{k,r}}_{t}}}{{{P^{D}_k}_{t}}}={{\omega^{Q}_{k,r}}}^{\frac{1}{{{\eta^{Q}_{k}}}}}\, \left(\frac{{{Q_{k,r}}_{t}}}{{{Q_k}_{t}}}\right)^{\frac{\left(-1\right)}{{{\eta^{Q}_{k}}}}}
\end{dmath}
% Equation 3
output production function
\begin{dmath}
{{Q_{k,r}}_{t}}= \left({{\omega^{Q^{I}}_{k,r}}}^{\frac{1}{{{\eta^{I}_{k,r}}}}}\, \left(Q^{I}_{k,r,t}\right)^{\frac{{{\eta^{Q^{I}}_{k,r}}}-1}{{{\eta^{Q^{I}}_{k,r}}}}}+\left(1 - {{\omega^{Q^{I}}_{k,r}}} \right)^{\frac{1}{{{\eta^{I}_{k,r}}}}} \, \left(Y_{k,r,t}\right)^{\frac{{{\eta^{I}_{k,r}}}-1}{{{\eta^{I}_{k,r}}}}}\right)^{\frac{{{\eta^{I}_{k,r}}}}{{{\eta^{I}_{k,r}}}-1}}
\end{dmath}
% Equation 4
demand for gross value added products
\begin{dmath}
\frac{{{P_{k,r}}_{t}}}{{P^{D}_{k,r,t}}}=\left(1 - {\omega^{Q^{I}}_{k,r}}\right)^{\frac{1}{{{\eta^{I}_{k}}}}} \, \left(\frac{Y_{k,r,t}}{Q_{k,r,t}}\right)^{\frac{\left(-1\right)}{{{\eta^{I}_{k}}}}}
\end{dmath}
% Equation 4
demand for intermediate products
\begin{dmath}
\frac{{P_{t}}}{{P^{D}_{k,r,t}}}={\omega^{Q^{I}}_{k,r}}^{\frac{1}{{{\eta^{I}_{k}}}}} \, \left(\frac{Q^{I}_{k,r,t}}{Q_{k,r,t}}\right)^{\frac{\left(-1\right)}{{{\eta^{I}_{k}}}}}
\end{dmath}

\subsection{Aggregation}
demand for domestically used sectoral products
\begin{align}
\frac{P^{D}_{k,t}}{P^{D}_{t}}={{\omega^{Q}_{k}}}^{\frac{1}{{{\eta^{Q}}}}}\, \left(\frac{Q^{D}_{k,t}}{{Q^{D}_{t}}}\right)^{\frac{\left(-1\right)}{{{\eta^{Q}}}}} \label{eq:demandsectoraloutput}
\end{align}
use of domestically produced products
\begin{align}
Q_{k,t} = Q^{D}_{k,t} + X_{k,t}
\end{align}
sectoral CES aggregation
\begin{align}
Q_{k,t}=\left(\sum_{r}^{R}{{\omega^{Q}_{k,r}}}^{\frac{1}{{{\eta^{Q}_{k}}}}}\, {{Q_{k,r}}_{t}}^{\frac{{{\eta^{Q}_{k}}}-1}{{{\eta^{Q}_{k}}}}}\right)^{\frac{{{\eta^{Q}_{k}}}}{{{\eta^{Q}_{k}}}-1}}
\end{align}
demand for domestically used and produced products
\begin{align}
\frac{P^{D}_{t}}{P_{t}}={(1 - \omega^{F})}^{\frac{1}{{{\eta^{F}}}}}\, \left(\frac{Q^{D}_{t}}{{Q^{U}_{t}}}\right)^{\frac{\left(-1\right)}{{{\eta^{F}}}}} \label{eq:demanddomesticoutput}
\end{align}
demand for imports
\begin{align}
\frac{P^{M}_{t}}{P_{t}}={(\omega^{F})}^{\frac{1}{{{\eta^{F}}}}}\, \left(\frac{M_{t}}{{Q^{U}_{t}}}\right)^{\frac{\left(-1\right)}{{{\eta^{F}}}}} \label{eq:demandimports}
\end{align}
demand for sectoral imports
\begin{align}
\frac{P^{M}_{k,t}}{P^{M}_{t}}={(\omega^{F})}^{\frac{1}{{{\eta^{F}}}}}\, \left(\frac{M_{k,t}}{{M_{t}}}\right)^{\frac{\left(-1\right)}{{{\eta^{F}}}}} \label{eq:demandsectoralimports}
\end{align}

\subsection{Aggregates}
resource constraint
\begin{align}
\frac{P^{D}_t}{P_t} \, Q_{t} = C_{t} + I_{t} + G_{t} + \sum_{k}^{K} \sum_{r}^{R} {G^{A}_{k,r,t}} + Q^{I}_{t} + {NX_{t}} \label{eq:resourceconstraint}
\end{align}
total demand for intermediate products
\begin{align}
P_{t} Q^{I}_{t} = \sum_{k}^{K} \sum_{r}^{R} P_{t} \, Q^{I}_{k,r,t}
\end{align}
total sectoral demand for intermediate products
\begin{align}
P_{t} Q^{I}_{k,t} = \sum_{r}^{R} P_{t} \, Q^{I}_{k,r,t}
\end{align}
definition of domestically used and produced products
\begin{align}
P^{D}_{t} \, Q^{D}_{t} = \sum_{k} P^{D}_{k,t} \, Q^{D}_{k,t}
\end{align}
definition of domestic output
\begin{align}
P_{t} \, Q_{t} = \sum_{k}^{K} P^{D}_{k,t} \, Q^{D}_{k,t}
\end{align}
definition of domestically used products
\begin{align}
P_{t} \, Q^{U}_{t} = P^{M}_{t} \, M_{t} + P^{D}_{t} \, Q^{D}_{t}
\end{align}
exports
\begin{align}
P^{D}_{t} X_{t} = \sum_{k}^{K} P^{D}_{k,t} \, X_{k,t}
\end{align}
imports
\begin{align}
P^{M}_{t} M_{t} = \sum_{k}^{K} P^{M}_{k,t} \, M_{k,t}
\end{align}
net exports
\begin{align}
P_{t} NX_{t} = P^{D}_{t} \, X_{t} - P^{M}_{t} \, M_{t} \label{eq:netexports}
\end{align}

\subsection{Trade}
demand for sectoral exports
\begin{align}
{X_{k,t}} = D^{X}_{k,t} \, Q_{k,t} \nonumber \\
D^{X}_{k,t} = D^{X}_{k,0} + \eta^{X}_{k,t} \label{eq:exportsdemand}
\end{align}
price for sectoral imports
\begin{align}
P^{M}_{k,t} = P^{M}_{k,0} + \eta^{M}_{k,t} \label{eq:importprices}
\end{align}


\section{Additional and modified model equations for energy extension}\label{app:modelequationsextension2}
\footnotesize
% Equation 1
\subsection{Regional Energy industries}
% Equation 3
output production function
\begin{dmath}
{{Q_{k,r}}_{t}}= \left({{\omega^{Q^{I}}_{k,r}}}^{\frac{1}{{{\eta^{I}_{k,r}}}}}\, \left(Q^{I,E}_{k,r,t}\right)^{\frac{{{\eta^{Q^{I}}_{k,r}}}-1}{{{\eta^{Q^{I}}_{k,r}}}}}+\left(1 - {{\omega^{Q^{I}}_{k,r}}} \right)^{\frac{1}{{{\eta^{I}_{k,r}}}}} \, \left(Y_{k,r,t}\right)^{\frac{{{\eta^{I}_{k,r}}}-1}{{{\eta^{I}_{k,r}}}}}\right)^{\frac{{{\eta^{I}_{k,r}}}}{{{\eta^{I}_{k,r}}}-1}}
\end{dmath}
nested composition function for intermediate energy and non-energy inputs
\begin{dmath}
{{Q^{I,E}_{k,r}}_{t}}= \left(({1-\omega^{Q^{E}}_{k,r}})^{\frac{1}{{{\eta^{E}_{k,r}}}}} \, \left(Q^{I,E}_{k,r,t}\right)^{\frac{{{\eta^{E}_{k,r}}}-1}{{{\eta^{E}}_{k,r}}}}+\left(\omega^{E}_{k,r} \right)^{\frac{1}{{{\eta^{E}_{k,r}}}}} \, \left(E_{k,r,t}\right)^{\frac{{{\eta^{E}_{k,r}}}-1}{{{\eta^{E}_{k,r}}}}}\right)^{\frac{{{\eta^{E}_{k,r}}}}{{{\eta^{E}_{k,r}}}-1}} \label{eq:energyintermediatecomposite}
\end{dmath}
nested composition function for energy inputs
\begin{dmath}
E_{k,r,t}= \left( \sum_{c}^{EC} {\omega^{Q^{E}}_{k,r,c}}^{\frac{1}{{{\eta^{EC}_{k,r}}}}}\, \left(E_{k,r,c,t}\right)^{\frac{{{\eta^{EC}_{k,r}}}-1}{{{\eta^{EC}}_{k,r}}}}\right)^{\frac{{{\eta^{EC}_{k,r}}}}{{{\eta^{EC}_{k,r}}}-1}} \label{eq:energycomposite}
\end{dmath}
demand for intermediate products
\begin{dmath}
\frac{{P^{I}_{k,r,t}}}{{P^{D}_{k,r,t}}}={\omega^{Q^{I}}_{k,r}}^{\frac{1}{{{\eta^{I}_{k}}}}} \, \left(\frac{Q^{I}_{k,r,t}}{Q^{I,E}_{k,r,t}}\right)^{\frac{\left(-1\right)}{{{\eta^{I}_{k}}}}}
\end{dmath}
demand for energy as intermediate product
\begin{dmath}
\frac{{P^{E}_{k,r,t}}}{{P^{I}_{k,r,t}}}={\omega^{E}_{k,r}}^{\frac{1}{{{\eta^{E}_{k,r}}}}} \, \left(\frac{E_{k,r,t}}{Q^{I,E}_{k,r,t}}\right)^{\frac{\left(-1\right)}{{{\eta^{E}_{k}}}}}
\end{dmath}
demand for non-energy intermediate products
\begin{dmath}
\frac{P_{t}}{P^{I}_{k,r,t}} = (1 - \omega^{E}_{k,r})^{\frac{1}{{{\eta^{E}_{k,r}}}}} \, \left(\frac{Q^{I}_{k,r,t}}{Q^{I,E}_{k,r,t}}\right)^{\frac{\left(-1\right)}{{{\eta^{E}_{k}}}}}
\end{dmath}


\subsection{Aggregates}
total demand for intermediate products
\begin{align}\label{eq:aggregationintermediateinputswithenergy}
P_{t} Q^{I}_{t} = \sum_{k}^{K} \sum_{r}^{R} \begin{cases}
			P^{I}_{k,r,t} \, Q^{I,E}_{k,r,t} & \mbox{if} k = k^{E} \\
			P_{t} \, Q^{I}_{k,r,t} & \mbox{otherwise}
			\end{cases}
\end{align}

total consumption of energy carrier
\begin{align}
E_{c,t} = \sum_{r}^{R} E_{k^{E},r,c,t} \label{eq:energycarrieraggregate}
\end{align}

price of energy carrier
\begin{align}
P^{EC}_{c,t} = P^{EC}_{c,0} + \eta^{EC}_{c,t} \label{eq:priceenergycarrier}
\end{align}



\cleardoublepage
\begin{center}
\begin{longtable}{lrr}
\caption{List of symbols for baseline version}\label{tab:listofsymbols}\\%
\hline
\multicolumn{1}{c}{\textbf{Variable}} &
\multicolumn{1}{c}{\textbf{\LaTeX}} &
\multicolumn{1}{c}{\textbf{Description}}\\%
\hline\hline%
\endfirsthead
\multicolumn{3}{c}{{\tablename} \thetable{} -- Continued}\\%
\hline%
\multicolumn{1}{c}{\textbf{Variable}} &
\multicolumn{1}{c}{\textbf{\LaTeX}} &
\multicolumn{1}{c}{\textbf{Description}}\\%
\hline\hline%
\endhead
\hline
\multicolumn{3}{c}{Endogenous}\\%
\hline%
\texttt{P} & $P$ & price level\\
\texttt{K} & $K$ & capital stock\\
\texttt{C} & $C$ & consumption\\
\texttt{PoP} & $Pop$ & population\\
\texttt{B} & $B$ & international traded bonds\\
\texttt{Sf} & $S^{f}$ & effective exchange rate with the rest of the world\\
\texttt{BG} & $BG$ & government debt\\
\texttt{NX} & $NX$ & net exports\\
\texttt{rf} & ${r^{f}}$ & foreign interest rate\\
\texttt{G} & $G$ & government expenditure\\
\texttt{I} & $I$ & private investment\\
\texttt{Y} & $Y$ & GDP\\
\texttt{N} & $N$ & labour\\
\texttt{SL} & ${SL}$ & sea level\\
\texttt{PREC\_k} & ${PREC_{r}}$ & regional PRECipitation\\
\texttt{T\_k} & ${T_{r}}$ & regional temperature\\
\texttt{WS\_k} & ${W_{r}^{S}}$ & regional wind speed\\
\texttt{Y\_k} & ${Y_k}$ & sector GDP\\
\texttt{K\_k} & ${K_k}$ & sector capital\\
\texttt{N\_k} & ${N_k}$ & sector employment\\
\texttt{I\_k} & ${I_k}$ & sector private investment\\
\texttt{P\_k} & ${P_k}$ & sector price index\\
\texttt{W\_k} & ${W_k}$ & sector wage index\\
\texttt{Y\_k\_r} & ${Y_{k,r}}$ & regional sector GDP\\
\texttt{D\_k\_r} & ${D_{k,r}}$ & regional sector damages\\
\texttt{K\_k\_r} & ${K_{k,r}}$ & regional sector capital\\
\texttt{N\_k\_r} & ${N_{k,r}}$ & regional sector employment\\
\texttt{W\_k\_r} & ${W_{k,r}}$ & regional sector wages\\
\texttt{A\_k\_r} & ${A_{k,r}}$ & regional sector TFP\\
\texttt{G\_A\_k\_r} & ${G^{A}_{k,r}}$ & regional sector adaptation government expenditure\\
\texttt{gA\_k\_r} & ${g^{A}_{k,r}}$ & regional growth rate of sector TFP\\
\texttt{A\_N\_k\_r} & ${A^{N}_{k,r}}$ & regional sector labour specific TFP\\
\texttt{A\_K\_k\_r} & ${A^{K}_{k,r}}$ & regional sector capital specific TFP\\
\texttt{I\_k\_r} & ${I_{k,r}}$ & regional sector private investment\\
\texttt{P\_k\_r} & ${P_{k,r}}$ & regional sector price index\\
\texttt{omegaI\_k\_r} & ${\omega^I_{k,r}}$ & shadow value of regional private sector investment\\
\texttt{r\_k\_r} & ${r_{k,r}}$ & regional sector rental rate on capital\\
\texttt{tauK\_k\_r} & ${\tau^{K}_{k,r}}$ & regional sector corporate tax rate on capital\\
\texttt{tauN\_k\_r} & ${\tau^{N}_{k,r}}$ & regional sector labour tax rate on capital\\
\hline
\multicolumn{3}{c}{Exogenous}\\%
\hline
\texttt{exo\_P} & ${\eta_{P}}$ & exogenous price index evolution\\
\texttt{exo\_PoP} & ${\eta_{Pop}}$ & exogenous population\\
\texttt{exo\_SL} & ${\eta_{SL}}$ & exogenous sea level\\
\texttt{exo\_NX} & ${\eta_{NX}}$ & exogenous net exports\\
\texttt{exo\_BG} & ${\eta_{BG}}$ & exogenous structural balance\\
\texttt{exo\_tauK\_k\_r} & ${\eta_{\tau^{K},k,r}}$ & exogenous sector and region corporate tax rate\\
\texttt{exo\_tauN\_k\_r} & ${\eta_{\tau^{N},k,r}}$ & exogenous sector and region labour tax rate\\
\texttt{exo\_k\_r} & ${\eta_{A,k,r}}$ & exogenous TFP\\
\texttt{exo\_N\_k\_r} & ${\eta_{A^{N},k,r}}$ & exogenous labour specific TFP\\
\texttt{exo\_K\_k\_r} & ${\eta_{A^{K},k,r}}$ & exogenous capital specific TFP\\
\texttt{exo\_GA\_k\_r} & ${\eta_{G^{A},k,r}}$ & exogenous sector adaptation expenditure\\
\texttt{exo\_T\_k} & ${\eta_{T,n}}$ & exogenous regional temperature\\
\texttt{exo\_PREC\_k} & ${\eta_{PREC,n}}$ & exogenous regional precipitation\\
\texttt{exo\_WS\_k} & ${\eta_{W^{S},n}}$ & exogenous regional wind speed\\
\hline%
\multicolumn{3}{c}{Parameters}\\%
\hline%
\texttt{omegaQ\_k\_p} & ${\omega^{Q}_{k}}$ & distribution parameter for output from one sector\\
\texttt{etaQ\_k\_p} & ${\eta^{C}_{k}}$ & elasticity of substitution between regional production\\
\texttt{tauK\_k\_r\_p} & ${\tau^{K}_{k,r}}$ & region and sector specific tax rate on capital\\
\texttt{tauN\_k\_r\_p} & ${\tau^{N}_{k,r}}$ & region and sector specific tax rate on labour\\
\texttt{rhoA\_k\_r\_p} & ${\rho^{A}_{k,r}}$ & persistence productivity shock\\
\texttt{rhoA\_N\_k\_r\_p} & ${\rho^{A,N}_{k,r}}$ & persistence labour specific productivity shock\\
\texttt{rhoA\_K\_k\_r\_p} & ${\rho^{A,K}_{k,r}}$ & persistence capital specific productivity shock\\
\texttt{phiY\_k\_r\_p} & $\frac{P_{k,r,0} \, Y_{k,r,0}}{P_{0} \, Y_{0}}$ & long-run share of regional and sectoral output\\
\texttt{phiN\_k\_r\_p} & ${N_{k,r,0}}$ & long run share of regional and sectoral employment\\
\texttt{phiY0\_k\_r\_p} & $\frac{P_{k,r,0} \, Y_{k,r,0}}{P_{0} \, Y_{0}}$ & terminal share of regional and sectoral output\\
\texttt{phiN0\_k\_r\_p} & ${N_{k,r,0}}$ & initial share of regional and sectoral employment\\
\texttt{phiYT\_k\_r\_p} & $\frac{P_{k,r,0} \, Y_{k,r,0}}{P_{0} \, Y_{0}}$ & terminal share of regional and sectoral output\\
\texttt{phiNT\_k\_r\_p} & ${N_{k,r,0}}$ & terminal share of regional and sectoral employment\\
\texttt{phiW\_k\_r\_p} & $\frac{W_{k,r,0} \, N_{k,r,0}}{P_{k,r,0} \, Y_{k,r,0}}$ & share of regional and sectoral employment\\
\texttt{phiP\_k\_r\_p} & $\frac{P_{k,r,0}}{P_{0}}$ & share of regional and sectoral employment\\
\texttt{phiL\_k\_r\_p} & ${\phi^{L}_{k,r}}$ & coefficient of disutility to work\\
\texttt{omegaQ\_k\_r\_p} & ${\omega^{Q}_{k,r}}$ & distribution parameter for regional production\\
\texttt{alphaK\_k\_r\_p} & ${\alpha^{K}_{k,r}}$ & distribution parameter capital share\\
\texttt{alphaN\_k\_r\_p} & ${\alpha^{N}_{k,r}}$ & distribution parameter labour share\\
\texttt{etaNK\_k\_r\_p} & ${\eta^{N,K}_{k,r}}$ & elasticity of substitution between labour and capital\\
\texttt{A\_k\_r\_p} & ${A_{k,r}}$ & sector long-run TFP\\
\texttt{GAT\_k\_r\_p} & ${G^{A}_{T,k,r}}$ & sector region specific government expenditure for adaptation\\
\texttt{phiGA\_k\_r\_p} & ${\phi^{G^{A}}_{k,r}}$ & coefficient of effectiveness of government expenditure on adaptation measures\\
\texttt{gY0\_k\_r\_p} & ${\frac{Y_{2,k,r}}{Y_{1,k,r}}}$ & initial sector growth\\
\texttt{gN0\_k\_r\_p} & $\frac{\frac{N_{2,k,r}}{N_{2}}}{\frac{N_{1,k,r}}{N_{1}}}$ & initial sector labour growth\\
\texttt{omegaA\_k\_r\_p} & ${\omega^{A}_{k,r}}$ & exponent for productivity growth\\
\texttt{A\_N\_k\_r\_p} & ${A^{N}_{k,r}}$ & sector labour specific TFP\\
\texttt{A\_K\_k\_r\_p} & ${A^{K}_{k,r}}$ & sector capital specific TFP\\
\texttt{a\_T\_k\_k\_r\_p} & ${a_{1,k,r}}$ & intercept of damage function temperature\\
\texttt{a\_T\_2\_k\_r\_p} & ${a_{2,k,r}}$ & slope of damage function temperature\\
\texttt{a\_T\_3\_k\_r\_p} & ${a_{3,k,r}}$ & exponent of damage function temperature\\
\texttt{a\_P\_k\_k\_r\_p} & ${a_{1,k,r}}$ & intercept of damage function PRECipitation\\
\texttt{a\_P\_2\_k\_r\_p} & ${a_{2,k,r}}$ & slope of damage function PRECipitation\\
\texttt{a\_P\_3\_k\_r\_p} & ${a_{3,k,r}}$ & exponent of damage function PRECipitation\\
\texttt{a\_W\_k\_k\_r\_p} & ${a_{1,k,r}}$ & intercept of damage function wind speed\\
\texttt{a\_W\_2\_k\_r\_p} & ${a_{2,k,r}}$ & slope of damage function wind speed\\
\texttt{a\_W\_3\_k\_r\_p} & ${a_{3,k,r}}$ & exponent of damage function wind speed\\
\texttt{a\_SL\_k\_k\_r\_p} & ${a_{1,k,r}}$ & intercept of damage function sea level\\
\texttt{a\_SL\_2\_k\_r\_p} & ${a_{2,k,r}}$ & slope of damage function sea level\\
\texttt{a\_SL\_3\_k\_r\_p} & ${a_{3,k,r}}$ & exponent of damage function sea level\\
\texttt{beta\_p} & ${\beta}$ & discount factor\\
\texttt{delta\_p} & ${\delta}$ & capital depreciation rate\\
\texttt{sigmaL\_p} & ${\sigma^{L}}$ & inverse Frisch elasticity\\
\texttt{sigmaC\_p} & ${\sigma^{C}}$ & intertemporal elasticity of substitution\\
\texttt{etaQ\_p} & ${\eta^{Q}}$ & elasticity of substitution between sectoral production\\
\texttt{phiB\_p} & ${\phi^{B}}$ & coefficient of foreign adjustment cost\\
\texttt{phiK\_p} & ${\phi^{K}}$ & coefficient of investment adjustment cost\\
\texttt{tauC\_p} & ${\tau^{C}}$ & consumption tax\\
\texttt{tauN\_p} & ${\tau^{N}}$ & labour tax\\
\texttt{tauK\_p} & ${\tau^{K}}$ & capital tax\\
\texttt{omegaNX\_p} & ${\omega^{NX}}$ & share of net exports relative to domestic GDP\\
\texttt{omegaNX0\_p} & ${\omega^{NX,0}}$ & initial share of net exports relative to domestic GDP\\
\texttt{omegaNXT\_p} & ${\omega^{NX,T}}$ & terminal share of net exports relative to domestic GDP\\
\texttt{rhoNX\_p} & ${\rho^{NX}}$ & persistency in net exports\\
\texttt{rhoA\_p} & ${\rho^{A}}$ & persistency in TFP\\
\texttt{rhoPoP\_p} & ${\rho^{Pop}}$ & persistency in population\\
\texttt{rhoSL\_p} & ${\rho^{SL}}$ & persistency in sea level\\
\texttt{rhoT\_p} & ${\rho^{T}}$ & persistency in temperature\\
\texttt{rhoWS\_p} & ${\rho^{T}}$ & persistency in wind speed\\
\texttt{rhoPREC\_p} & ${\rho^{T}}$ & persistency in PRECipitation\\
\texttt{inbsectors\_p} & ${K}$ & number of sectors\\
\texttt{inbregions\_p} & ${R}$ & number of regions\\
\texttt{lCalibration\_p} & ${l^{Calib}}$ & logical indiactor whether model is calibrated or not\\
\texttt{T0\_r\_p} & ${T_{0,n}}$ & initial regional temperature\\
\texttt{PREC0\_r\_p} & ${PREC_{0,n}}$ & initial regional PRECipitation\\
\texttt{WS0\_r\_p} & ${W^{S}_{0,n}}$ & initiial regional wind speed\\
\texttt{TT\_r\_p} & ${T_{T,n}}$ & terminal regional temperature\\
\texttt{PRECT\_r\_p} & ${PREC_{T,n}}$ & terminal regional PRECipitation\\
\texttt{WST\_r\_p} & ${W^{S}_{T,n}}$ & terminal regional wind speed\\
\texttt{SL0\_p} & ${SL_0}$ & initial sea level\\
\texttt{PoP0\_p} & ${POP_0}$ & initial population\\
\texttt{Y0\_p} & ${Y_0}$ & initial output\\
\texttt{P0\_p} & ${P_0}$ & initial price level\\
\texttt{N0\_p} & ${N_0}$ & initial employment\\
\texttt{SLT\_p} & ${SL_T}$ & terminal sea level\\
\texttt{PoPT\_p} & ${Pop_T}$ & terminal population\\
\texttt{YT\_p} & ${Y_0}$ & terminal output\\
\texttt{NT\_p} & ${Y_0}$ & terminal employment\\
\hline%
\end{longtable}
\end{center}


\begin{center}
\begin{longtable}{lrr}
\caption{List of symbols for input-output extension}\label{tab:listofsymbolsextension1}\\%
\hline%
\multicolumn{1}{c}{\textbf{Variable}} &
\multicolumn{1}{c}{\textbf{\LaTeX}} &
\multicolumn{1}{c}{\textbf{Description}}\\%
\hline\hline%
\endfirsthead
\multicolumn{3}{c}{{\tablename} \thetable{} -- Continued}\\%
\hline%
\multicolumn{1}{c}{\textbf{Variable}} &
\multicolumn{1}{c}{\textbf{\LaTeX}} &
\multicolumn{1}{c}{\textbf{Description}}\\%
\hline\hline%
\endhead
\hline
\multicolumn{3}{c}{Endogenous}\\%
\hline
\texttt{P\_D} & $P^{D}$ & price level of domestic produced products\\
\texttt{P\_M} & $P^{M}$ & price level of imports\\
\texttt{Q\_U} & $Q^{U}$ & domestic used products \\
\texttt{Q\_D} & $Q^{D}$ & domestic produced and used products\\
\texttt{Q\_I} & $Q^{I}$ & intermediate products\\
\texttt{Q} & $Q$ & doesmtic output\\
\texttt{M} & $M$ & imports\\
\texttt{X} & $X$ & exports\\
\texttt{Q\_k} & ${Q_{k}}$ & sectoral exports\\
\texttt{Q\_D\_k} & ${Q^{D}_{k}}$ & domestic used and produced products\\
\texttt{Q\_I\_k} & ${Q^{I}_{k}}$ & domestic used intermediate products\\
\texttt{M\_k} & ${M_{k}}$ & sectoral imports\\
\texttt{X\_k} & ${X_{k}}$ & sectoral exports\\
\texttt{P\_D\_k} & ${P^{D}_{k}}$ & sectoral price index of domestic products\\
\texttt{P\_M\_k} & ${P^{D}_{k}}$ & sectoral price index of imports\\
\texttt{Q\_k\_r} & ${Q_{k,r}}$ & regional sector output\\
\texttt{Q\_I\_k\_r} & ${Q_{k,r}}$ & regional sector demand for intermediate products\\
\texttt{P\_D\_k\_r} & ${P^{D}_{k,r}}$ & sectoral and regional price index of domestic products\\
\hline
\multicolumn{3}{c}{Exogenous}\\%
\hline
\texttt{exo\_X\_k} & ${\eta^{X}_{k}}$ & exogenous demand for sectoral exports\\
\texttt{exo\_M\_k} & ${\eta^{M}_{k}}$ & exogenous sectoral price of imports\\
\hline
\multicolumn{3}{c}{Parameters}\\%
\hline
\texttt{phiM\_k\_p} & ${\phi^{M}_{k}}$ & initial share of sectoral imports\\
\texttt{phiX\_k\_p} & ${\phi^{X}_{k}}$ & initial share of sectoral exports\\
\texttt{phiQI\_k\_p} & ${\phi^{Q^{I}}_{k}}$ & initial share of sectoral intermediate inputs\\
\texttt{D\_X\_k\_p} & ${D^{X}_{k}}$ & initial demand for sectoral exports\\
\texttt{P\_M\_k\_p} & ${P^{M}_{k}}$ & initial price level for sectoral imports\\
\texttt{omega\_M\_k\_p} & ${\omega^{M}_{k}}$ & distribution parameter for imports\\
\texttt{eta\_I\_k\_p} & ${\eta^{I}_{k}}$ & elasticity of substitution between intermediate products and value added\\
\texttt{omegaQI\_k\_r\_p} & ${\omega^{Q^{I}}_{k,r}}$ & distribution parameter between intermediate products and value added\\
\texttt{etaM\_p} & ${\eta^{M}}$ & elasticity of substitution between sectoral imports\\
\texttt{etaF\_p} & ${\eta^{F}}$ & elasticity of substitution between imports and domestic products\\
\texttt{omegaF\_p} & ${\omega^{F}}$ & distribution parameter for imports\\
\texttt{phiM\_p} & ${\phi^{M}}$ & initial share of imports on toal output\\
\hline%
\end{longtable}
\end{center}


\begin{center}
\begin{longtable}{lrr}
\caption{List of symbols for energy extension}\label{tab:listofsymbolsextension2}\\%
\hline%
\multicolumn{1}{c}{\textbf{Variable}} &
\multicolumn{1}{c}{\textbf{\LaTeX}} &
\multicolumn{1}{c}{\textbf{Description}}\\%
\hline\hline%
\endfirsthead
\multicolumn{3}{c}{{\tablename} \thetable{} -- Continued}\\%
\hline%
\multicolumn{1}{c}{\textbf{Variable}} &
\multicolumn{1}{c}{\textbf{\LaTeX}} &
\multicolumn{1}{c}{\textbf{Description}}\\%
\hline\hline%
\endhead
\hline
\multicolumn{3}{c}{Endogenous}\\%
\hline
\texttt{P\_EC\_c} & $P^{E}_{c}$ & price of energy source\\
\texttt{EC\_c} & $E_{c}$ & consumption of energy source\\
\texttt{E\_k\_r} & $E_{k,r}$ & composite energy consumption \\
\texttt{Q\_IE\_k\_r} & $Q^{I,E}_{k,r}$ & composite intermediate products for energy sector \\
\texttt{P\_I\_k\_r} & $P^{E,Q^{I}}_{k,r}$ & price of composite intermediate products for energy sector \\
\texttt{P\_E\_k\_r} & $P^{E}_{k,r}$ & price of composite energy \\
\texttt{E\_k\_r\_c} & $E_{k,r,c}$ & regional and sectoral consumption of energy source \\
\hline
\multicolumn{3}{c}{Exogenous}\\%
\hline
\texttt{exo\_E\_c} & ${\eta^{E}_{c}}$ & exogenous price for energy source\\
\hline
\multicolumn{3}{c}{Parameters}\\%
\hline
\texttt{phiE\_k\_r\_p} & ${\phi^{E}_{k,r}}$ & initial share of energy on intermediate products\\
\texttt{phiE\_k\_\_r\_c\_p} & ${\phi^{E}_{k,r,c}}$ & initial share of energy source on energy composite intermediate product\\
\texttt{omegaE\_k\_r\_p} & ${\phi^{E}_{k,r}}$ & distribution parameter for energy on intermediate products\\
\texttt{omegaE\_k\_\_r\_c\_p} & ${\phi^{E}_{k,r,c}}$ & distribution parameter for energy source on energy as intermediate product\\
\texttt{etaE\_k\_r\_p} & ${\eta^{E}_{k,r}}$ & elasticity of substitution between energy and other intermediate products\\
\texttt{etaEC\_k\_r\_p} & ${\eta^{E,C}_{k,r}}$ & elasticity of substitution between energy carriers\\
\texttt{P\_EC\_c\_p} & ${P^{E}_{c,0}}$ & initial price for energy source\\
\hline%
\end{longtable}
\end{center}

\end{document}